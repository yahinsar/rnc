\documentclass[bachelor, och, coursework]{shiza}
% параметр - тип обучения - одно из значений:
%    spec     - специальность
%    bachelor - бакалавриат (по умолчанию)
%    master   - магистратура
% параметр - форма обучения - одно из значений:
%    och   - очное (по умолчанию)
%    zaoch - заочное
% параметр - тип работы - одно из значений:
%    referat    - реферат
%    coursework - курсовая работа (по умолчанию)
%    diploma    - дипломная работа
%    pract      - отчет по практике
% параметр - включение шрифта
%    times    - включение шрифта Times New Roman (если установлен)
%               по умолчанию выключен
\usepackage{subfigure}
\usepackage{tikz,pgfplots}
\pgfplotsset{compat=1.5}
\usepackage{float}
\usepackage{pdfpages}

%\usepackage{titlesec}
\setcounter{secnumdepth}{4}
%\titleformat{\paragraph}
%{\normalfont\normalsize}{\theparagraph}{1em}{}
%\titlespacing*{\paragraph}
%{35.5pt}{3.25ex plus 1ex minus .2ex}{1.5ex plus .2ex}

\titleformat{\paragraph}[block]
{\hspace{1.25cm}\normalfont}
{\theparagraph}{1ex}{}
\titlespacing{\paragraph}
{0cm}{2ex plus 1ex minus .2ex}{.4ex plus.2ex}

% --------------------------------------------------------------------------%


\usepackage[T2A]{fontenc}
\usepackage[utf8]{inputenc}
\usepackage{graphicx}
\graphicspath{ {./images/} }
\usepackage{tempora}

\usepackage[sort,compress]{cite}
\usepackage{amsmath}
\usepackage{amssymb}
\usepackage{amsthm}
\usepackage{fancyvrb}
\usepackage{listings}
\usepackage{listingsutf8}
\usepackage{longtable}
\usepackage{array}
\usepackage[english,russian]{babel}

%\usepackage[colorlinks=true]{hyperref}
\usepackage{url}

\usepackage{underscore}
\usepackage{setspace}
\usepackage{indentfirst} 
\usepackage{mathtools}
\usepackage{amsfonts}
\usepackage{enumitem}
\usepackage{tikz}

\usepackage{minted}
\setminted{style=vs,autogobble=true,fontsize=\footnotesize,breaklines=true}
\newcommand{\eqdef}{\stackrel {\rm def}{=}}
\newcommand{\specialcell}[2][c]{%
	\begin{tabular}[#1]{@{}c@{}}#2\end{tabular}}

\renewcommand\theFancyVerbLine{\small\arabic{FancyVerbLine}}

\newtheorem{lem}{Лемма}

\lstset{%
	language=C,
	basicstyle=\ttfamily\fontsize{11pt}{11pt}\selectfont,
	%basicstyle=\footnotesize\ttfamily,
	breaklines=true,
	%backgroundcolor=\color{gray!12},
	basicstyle=\ttfamily\small,
	keywordstyle=\color{blue},
	commentstyle= \color{green},
	stringstyle=\color{orange},
	%identifierstyle=\color{yellow},
	showstringspaces=false,
	captionpos=b,
	%numbers=left,
	%numberstyle=\footnotesize\color{gray},
	%frame=TB,
	tabsize=2,
	%keepspaces=true,
	morekeywords={procedure, then, begin, end}
}


\begin{document}
	
	% Кафедра (в родительном падеже)
	%\chair{теоритических основ компьютерной безопасности и криптографии}
	
	% Тема работы
	%\title{Разделение секрета с помощью схемы Шамира}
	
	% Курс
	%\course{3}
	
	% Группа
	%\group{331}
	
	% Факультет (в родительном падеже) (по умолчанию "факультета КНиИТ")
	%\department{факультета КНиИТ}
	
	% Специальность/направление код - наименование
	%\napravlenie{09.03.04 "--- Программная инженерия}
	%\napravlenie{010500 "--- Математическое обеспечение и администрирование информационных систем}
	%\napravlenie{230100 "--- Информатика и вычислительная техника}
	%\napravlenie{231000 "--- Программная инженерия}
	%\napravlenie{100501 "--- Компьютерная безопасность}
	
	% Для студентки. Для работы студента следующая команда не нужна.
	% \studenttitle{Студентки}
	
	% Фамилия, имя, отчество в родительном падеже
	%\author{Яхина Шамиля Илдусовича}
	
	% Заведующий кафедрой
	%\chtitle{доцент, к.ф.-м.н.} % степень, звание
	%\chname{М.~Б.~Абросимов}
	
	%Научный руководитель (для реферата преподаватель проверяющий работу)
	%\satitle{ассистент} %должность, степень, звание
	%\saname{А. А. Лобов}
	
	% Руководитель практики от организации (только для практики,
	% для остальных типов работ не используется)
	%\patitle{к.ф.-м.н.}
	%\paname{М.~Б.~Абросимов}
	
	% Семестр (только для практики, для остальных
	% типов работ не используется)
	%\term{8}
	
	% Наименование практики (только для практики, для остальных
	% типов работ не используется)
	%\practtype{преддипломная}
	
	% Продолжительность практики (количество недель) (только для практики,
	% для остальных типов работ не используется)
	%\duration{4}
	
	% Даты начала и окончания практики (только для практики, для остальных
	% типов работ не используется)
	%\practStart{30.04.2019}
	%\practFinish{27.05.2019}
	
	% Год выполнения отчета
	%\date{2022}
	
	%\maketitle
	
	% Включение нумерации рисунков, формул и таблиц по разделам
	% (по умолчанию - нумерация сквозная)
	% (допускается оба вида нумерации)
	% \secNumbering
	\includepdf{title-3.pdf}
	
	%-------------------------------------------------------------------------------------------
	%\tableofcontents
	%\includepdf{table-of-content.pdf}
	%ПЛАН ДЛЯ КУРСОВОЙ:
		
	\section{Генератор псевдослучайных чисел}
	
	Описание задания:
	
	Создайте программу для генерации псевдослучайных величин следующими алгоритмами:
	\begin{enumerate}
		\item Линейный конгруэнтный метод;
		\item Аддитивный метод;
		\item Пятипараметрический метод;
		\item Регистр сдвига с обратной связью (РСЛОС);
		\item Нелинейная комбинация РСЛОС;
		\item Вихрь Мерсенна;
		\item RC4;
		\item ГПСЧ на основе RSA;
		\item Алгоритм Блюма-Блюма-Шуба.
	\end{enumerate}
	
	
	\subsection{Линейный конгруэнтный метод}
	
	Описание алгоритма:
	
	Последовательность ПСЧ, получаемая по формуле:
	\begin{center}
		$X_{n + 1} = (aX_n + c) \mod m, \; n \geq 1,$
	\end{center}
	называется \textit{линейной конгруэнтной последовательностью (ЛКП)}.
	
	В его основе лежит выбор четырех ключевых чисел:
	\begin{itemize}
		\item $m > 0$, модуль;
		\item $0 \leq a \leq m$, множитель;
		\item $0 \leq c \leq m$, приращение;
		\item $0 \leq X_0 \leq m$, начальное значение.
	\end{itemize}

	Инициализирующий вектор: модуль, множитель, приращение, начальное значение.
	
	Параметры запуска программы:
	
	\textit{prng.exe /n:16000 /g:lc /i:12960,1741,2731,1 /f:prngTestLc.txt}

	\begin{figure}[H]
	\centering
	\includegraphics[width=0.9\textwidth]{lcCmd}
	\caption{Запуск программы}
	\label{fig:lcCmd}
	\end{figure}

	\begin{figure}[H]
	\centering
	\includegraphics[width=0.9\textwidth]{lcTest}
	\caption{Сгенерированная последовательность}
	\label{fig:lcTest}
	\end{figure}

	Исходный код программы:
	
	\begin{minted}[fontsize=\footnotesize]{cpp}
		void lcFunc(string codeMethod, vector <int> genVec, int NumCount, const char* fileName, bool defaultVec) {
			long long m = 12960;
			long long a = 1741;
			long long c = 2731;
			long long x0 = 1;
			if (!defaultVec) {
				m = genVec[0];
				a = genVec[1];
				c = genVec[2];
				x0 = genVec[3];
			}
			if (m <= 0 || a > m || a < 0 || c > m || c < 0 || x0 > m || x0 < 0) {
				cout << "\nError. The parameters do not meet the requirements.\n";
				return;
			}
			if (nod(c, m) != 1) {
				cout << "\nThe parameters do not meet the requirements of Theorem 3.1 (Item 1).\n";
			}
			if (!(kratP(a - 1, m))) {
				cout << "\nThe parameters do not meet the requirements of Theorem 3.1 (Item 2).\n";
			}
			if (m % 4 == 0 && (a - 1) % 4 != 0) {
			cout << "\nThe parameters do not meet the requirements of Theorem 3.1 (Item 3).\n";
		}
		vector<int> showsVector = progressVector(NumCount);
		int check_progress_index = 0;
		
		ofstream f;
		f.open(fileName, ios::out);
		for (int i = 0; i < NumCount; i++)
		{
			x0 = (x0 * a + c) % m;
			if (defaultVec)
			f << x0 % 1001 << ",";
			else
			f << x0 << ",";
			
			if (i == showsVector[check_progress_index]) {
				cout << (check_progress_index + 1) * 10 << "% completed\n";
				check_progress_index++;
			}
		}
		f.close();
		return;
	}
	\end{minted}
	
	\subsection{Аддитивный метод}
	
	Описание алгоритма:
	
	Последовательность определяется следующим образом:
	\begin{center}
		$X_n = (X_{n - k} + X_{n - j}) \mod m, \; j > k \geq 1$
	\end{center}

	Инициализирующий вектор: модуль, младший индекс, старший индекс, последовательность начальных значений.
	
	Параметры запуска программы:
	
	\textit{prng.exe /n:16000 /g:add /i:8001,18,65,816,159,798,290,168,441,691,655,874, 220,125,977,586,381,868,294,948,437,581,181,701,536,11,672,103,601,794,189,12,130, 386,828,288,183,117,456,624,807,110,498,27,234,474,613,615,341,906,562,778,486, 155,276,894,441,226,762,234,762,98,458,399,445,765,223,879 /f:prngTestAdd.txt}
	
	\begin{figure}[H]
		\centering
		\includegraphics[width=0.9\textwidth]{addCmd}
		\caption{Запуск программы}
		\label{fig:addCmd}
	\end{figure}
	
	\begin{figure}[H]
		\centering
		\includegraphics[width=0.9\textwidth]{addTest}
		\caption{Сгенерированная последовательность}
		\label{fig:addTest}
	\end{figure}

	Исходный код программы:
	
	\begin{minted}[fontsize=\footnotesize]{cpp}
	void addFunc(string codeMethod, vector <int> genVec, int NumCount, const char* fileName, bool defaultVec) {
		// Input: module, low index, high index, sequence of initial values
		long long m = 8001;
		long long k = 18;
		long long j = 65;
		long long x0;
		vector <long long> xNs;
		if (!defaultVec) {
			m = genVec[0];
			k = genVec[1];
			j = genVec[2];
			for (int i = 3; i < genVec.size(); i++)
			{
				xNs.push_back(genVec[i]);
			}
		}
		else {
			xNs = { 816, 159, 798, 290, 168, 441, 691, 655, 874, 220, 125, 977, 586, 381, 868, 294, 948, 437, 581, 181, 701, 536, 11, 672, 103, 601, 794, 189, 12, 130, 386, 828, 288, 183, 117, 456, 624, 807, 110, 498, 27, 234, 474, 613, 615, 341, 906, 562, 778, 486, 155, 276, 894, 441, 226, 762, 234, 762, 98, 458, 399, 445, 765, 223, 879 }
			;
		}
		
		if (k < 1 || k >= j || j < 1) {
			cout << "\nError. The parameters do not meet the requirements.\n";
			return;
		}
		
		vector<int> showsVector = progressVector(NumCount);
		int check_progress_index = 0;
		
		ofstream f;
		f.open(fileName, ios::out);
		long long maxkj = max(k, j);
		if (xNs.size() < maxkj) {
			cout << "\nError. The parameters do not meet the requirements (The transferred initial values are insufficient).\n";
			return;
		}
		for (int i = maxkj + 1; i < NumCount + maxkj + 1; i++)
		{
			x0 = (xNs[i - k] + xNs[i - j]) % m;
			xNs.push_back(x0);
			if (defaultVec)
			f << x0 % 1001 << ",";
			else
			f << x0 << ",";
			
			if (i == showsVector[check_progress_index]) {
				cout << (check_progress_index + 1) * 10 << "% completed\n";
				check_progress_index++;
			}
		}
		f.close();
		return;
	}
	\end{minted}
	
	\subsection{Пятипараметрический метод}
	
	Описание алгоритма:
	
	Данный метод является частным случаем РСЛОС, использует характеристический многочлен из 5 членов и позволяет генерировать последовательности $w$-битовых двоичных целых чисел в соответствии со следующей рекуррентной 
	формулой:
	\begin{center}
		$X_{n + p} = X_{n + q_1} + X_{n + q_2} + X_{n + q_3} + X_n, \; n = 1, 2, 3, \dots$
	\end{center}
	
	Инициализирующий вектор: $p, q_1, q_2, q_3, w$.
	
	Параметры запуска программы:
	
	\textit{prng.exe /n:16000 /g:5p /i:4253,1093,2254,3297,16 /f:prngTest5p.txt}
	
	\begin{figure}[H]
		\centering
		\includegraphics[width=0.9\textwidth]{5pCmd}
		\caption{Запуск программы}
		\label{fig:5pCmd}
	\end{figure}
	
	\begin{figure}[H]
		\centering
		\includegraphics[width=0.9\textwidth]{5pTest}
		\caption{Сгенерированная последовательность}
		\label{fig:5pTest}
	\end{figure}
	
		Исходный код программы:
	
	\begin{minted}[fontsize=\footnotesize]{cpp}
		void fivePFunc(string codeMethod, vector <int> genVec, int NumCount, const char* fileName, bool defaultVec) {
			// Input: p, q1, q2, q3, w
			int p = 4253;
			long long q1 = 1093;
			long long q2 = 2254;
			long long q3 = 3297;
			int w = 16;
			if (!defaultVec) {
				p = genVec[0];
				q1 = genVec[1];
				q2 = genVec[2];
				q3 = genVec[3];
				w = genVec[4];
			}
			
			if (q1 >= p || q2 >= p || q3 >= p) {
				cout << "\nError. The parameters do not meet the requirements.\n";
				return;
			}
			
			int regLength = p;
			deque <bool> reg = gen_p_lc(p);
			vector<int> showsVector = progressVector(NumCount);
			int check_progress_index = 0;
			
			ofstream f;
			f.open(fileName, ios::out);
			for (int i = 0; i < NumCount; i++)
			{
				bool xNplusP = reg[regLength - q1 - 1] ^ reg[regLength - q2 - 1] ^ reg[regLength - q3 - 1];
				reg.push_front(xNplusP);
				reg.pop_back();
				long long ReginDec = 0;
				for (int h = 0; h < w; h++)
				{
					ReginDec *= 2;
					ReginDec += reg[h];
				}
				if (defaultVec)
				f << ReginDec % 1001 << ",";
				else
				f << ReginDec << ",";
				
				if (i == showsVector[check_progress_index]) {
					cout << (check_progress_index + 1) * 10 << "% completed\n";
					check_progress_index++;
				}
			}
			f.close();
			return;
		}
	\end{minted}
	
	
	\subsection{Регистр сдвига с обратной связью (РСЛОС)}
	
	Описание алгоритма:
	
	Для натурального числа $p$ и $c_1,c_2,…,c_(p-1)$, принимающих значения 0 или 1 определяют рекуррентную формулу
	\begin{center}
		$X_{n+p}  = a_{p-1} X_{n+p-1}  + a_{p-2} X_{n+p-2}  + … + a_1 X_{n+1}  + X_n$
	\end{center}
	
	Одна итерация алгоритма, генерирующего последовательность, состоит 
	из следующих шагов:
	\begin{enumerate}
		\item Содержимое ячейки $p - 1$ формирует очередной бит ПСП битов.
		\item Содержимое ячейки 0 определяется значением функции обратной связи, 
		являющейся линейной булевой функцией с коэффициентами $a_1, \dots, a_{p - 1}$.
		\item Содержимое каждого $i$-го бита перемещается в $(i + 1)$-й, $0 \leq i < p - 1$. 
		\item В ячейку 0 записывается новое содержимое, вычисленное на шаге 2.
	\end{enumerate} \

	Инициализирующий вектор: двоичное представление вектора коэффициентов, начальное значение регистра.
	
	Параметры запуска программы:
	
	\textit{prng.exe /n:16000 /g:lfsr /i:100000000101011,121 /f:prngTestLfsr.txt}

	\begin{figure}[H]
		\centering
		\includegraphics[width=0.9\textwidth]{lfsrCmd}
		\caption{Запуск программы}
		\label{fig:lfsrCmd}
	\end{figure}
	
	\begin{figure}[H]
		\centering
		\includegraphics[width=0.9\textwidth]{lfsrTest}
		\caption{Сгенерированная последовательность}
		\label{fig:lfsrTest}
	\end{figure}

		Исходный код программы:
	
	\begin{minted}[fontsize=\footnotesize]{cpp}
		void lfsrFunc(string codeMethod, vector <int> genVec, int NumCount, const char* fileName, bool defaultVec) {
			// Input: binary representation of the coefficient vector, initial value of the register
			if (defaultVec) {
				genVec = { 121, 14, 5, 3, 1, 0 }; // x0, coeffs (100000000101011)
			}
			int x0 = genVec[0];
			int x0Length = log2(x0) + 1;
			int regLength = x0Length - 1;
			bool bigCoeff = false;
			for (int i = 1; i < genVec.size(); i++)
			{
				if (genVec[i] > regLength) {
					regLength = genVec[i] + 1;
					bigCoeff = true;
				}
			}
			if (x0 < 0) {
				cout << "\nError. The parameters do not meet the requirements.\n";
				return;
			}
			
			vector<int> showsVector = progressVector(NumCount);
			int check_progress_index = 0;
			
			deque<bool> reg;
			if (!bigCoeff) {
				regLength++;
				reg = binary(x0, regLength);
			}
			else {
				reg.assign(regLength - x0Length, 0);
				deque<bool> regDop = binary(x0, x0Length);
				for (int i = 0; i < regDop.size(); i++)
				{
					reg.push_back(regDop[i]);
				}
			}
			ofstream f;
			f.open(fileName, ios::out);
			for (int i = 0; i < NumCount; i++)
			{
				bool xNplusP = reg[regLength - genVec[1] - 1];
				for (int j = 2; j < genVec.size(); j++)
				{
					xNplusP ^= reg[regLength - genVec[j] - 1];
				}
				reg.push_front(xNplusP);
				reg.pop_back();
				long long ReginDec = 0;
				for (int h = 0; h < reg.size(); h++)
				{
					ReginDec *= 2;
					ReginDec += reg[h];
				}
				if (defaultVec)
				f << ReginDec % 1001 << ",";
				else
				f << ReginDec << ",";
				
				if (i == showsVector[check_progress_index]) {
					cout << (check_progress_index + 1) * 10 << "% completed\n";
					check_progress_index++;
				}
			}
			f.close();
			return;
		}
	\end{minted}
	

	\subsection{Нелинейная комбинация РСЛОС}
	
	Описание алгоритма:
	
	Последовательность получается объединением нелинейным образом трёх РСЛОС. Нелинейная функция генератора:
	\begin{center}
		$f(x_1, x_2, x_3) = x_1 x_2 \oplus (1 + x_2) x_3 = x_1 x_2 \oplus x_2 x_3 \oplus x_3$
	\end{center}
	
	Инициализирующий вектор: двоичное представление векторов коэффициентов для $R1$, $R2$, $R3$.
	
	Параметры запуска программы:
	
	\textit{prng.exe /n:16000 /g:nfsr /i:100000000011011,1000000000000011,} 
	
	\textit{10000000000101101 /f:prngTestNfsr.txt}

	\begin{figure}[H]
		\centering
		\includegraphics[width=0.9\textwidth]{nfsrCmd}
		\caption{Запуск программы}
		\label{fig:nfsrCmd}
	\end{figure}
	
	\begin{figure}[H]
		\centering
		\includegraphics[width=0.9\textwidth]{nfsrTest}
		\caption{Сгенерированная последовательность}
		\label{fig:nfsrTest}
	\end{figure}

		Исходный код программы:
	
	\begin{minted}[fontsize=\footnotesize]{cpp}
		void nfsrFunc(string codeMethod, vector <int> genVec, int NumCount, const char* fileName, vector <deque <bool>> nfsrRegs, bool defaultVec) {
			// Input: binary representation of the coefficient vectors for R1, R2, R3
			if (defaultVec) {
				genVec = { 5, 14, 4, 3, 1, 0, 3, 15, 1, 0, 5, 16, 5, 3, 2, 0 }; // size1,coeffs1,size2,coeffs2,size3,coeffs3 (100000000011011,1000000000000011,10000000000101101)
				nfsrRegs = { {1,0,0,0,0,0,0,0,0,0,1,1,0,1,1}, {1,0,0,0,0,0,0,0,0,0,0,0,0,0,1,1}, {1,0,0,0,0,0,0,0,0,0,0,1,0,1,1,0,1} };
			}
			int nachR1 = 1;
			int endR1 = genVec[0] + 1;
			int nachR2 = endR1 + 1;
			int endR2 = genVec[endR1] + nachR2 + 1;
			int nachR3 = endR2 + 1;
			int endR3 = genVec.size();
			vector<int> startsRslos = { nachR1, nachR2, nachR3 };
			vector<int> endssRslos = { endR1, endR2, endR3 };
			vector<int> regLengths;
			for (int i = 0; i < 3; i++)
			{
				regLengths.push_back(nfsrRegs[i].size());
			}
			vector <deque<bool>> regs = nfsrRegs;
			
			vector<int> showsVector = progressVector(NumCount);
			int check_progress_index = 0;
			
			ofstream f;
			f.open(fileName, ios::out);
			for (int i = 0; i < NumCount; i++)
			{
				for (int j = 0; j < 3; j++)
				{
					int startH = startsRslos[j];
					int endH = endssRslos[j];
					bool xNext = regs[j][regLengths[j] - genVec[startH] - 1];
					for (int h = startH; h < endH; h++)
					{
						xNext ^= regs[j][regLengths[j] - genVec[h] - 1];
					}
					regs[j].push_front(xNext);
					regs[j].pop_back();
				}
				vector <deque<bool>> regs2(2);
				regs2[0] = multRs(regs[0], regs[1]);
				regs2[1] = multRs(regs[1], regs[2]);
				deque<bool> resR = xorRs(xorRs(regs2[0], regs2[1]), regs[2]);
				
				long long ReginDec = 0;
				for (int h = 0; h < resR.size(); h++)
				{
					ReginDec *= 2;
					ReginDec += resR[h];
				}
				if (defaultVec)
				f << ReginDec % 1001 << ",";
				else
				f << ReginDec << ",";
				
				if (i == showsVector[check_progress_index]) {
					cout << (check_progress_index + 1) * 10 << "% completed\n";
					check_progress_index++;
				}
			}
			f.close();
			return;
		}
	\end{minted}
	

	\subsection{Вихрь Мерсенна}
	
	Описание алгоритма:
	
	Метод Вихрь Мерсенна позволяет генерировать последовательность двоичных псевдослучайных целых w-битных чисел в соответствии с рекуррентной формулой:
	\begin{center}
	$X_{n + p} = X_{n + q} \oplus (X_n^r | X_{n + 1}^l )A$ $ (n = 0,1,2, \dots),$
	\end{center}

	где $p, q, r$ – целые константы, $p$ – степень реккурентности, $1 \leq q \leq p$;
	
	$X_n$ - $w$-битное двоичное целое число;
	
	$(X_n^r | X_{n+1}^l )$ – двоичное целое число, полученное конкатенацией чисел $X_n^r$ и $X_{n+1}^l$, когда первые $(w-r)$ битов взяты из $X_n$, а последние $r$ битов из $X_{n+1}$ в том же порядке;
	
	$A$ – матрица размера $w \times w$  состоящая из нулей и единиц, определенная посредством a;
	
	$XA$ – произведение, при вычислении которого сначала выполняют операцию $X >> 1$, если последний бит $X$ равен 0, а затем, когда последний бит $X$ равен 1, то вычисляют $XA=(X >> 1)  \oplus A$	\\
	
	Инициализирующий вектор: модуль, начальное значение $x$.
	
	Параметры запуска программы:
	
	\textit{prng.exe /n:16000 /g:mt /i:10001,8191 /f:prngTestMt.txt}
	
	\begin{figure}[H]
	\centering
		\includegraphics[width=0.9\textwidth]{mtCmd}
		\caption{Запуск программы}
		\label{fig:mtCmd}
	\end{figure}
	
	\begin{figure}[H]
		\centering
		\includegraphics[width=0.9\textwidth]{mtTest}
		\caption{Сгенерированная последовательность}
		\label{fig:mtTest}
	\end{figure}	
	
		Исходный код программы:
	
	\begin{minted}[fontsize=\footnotesize]{cpp}
		void mtFunc(string codeMethod, vector <int> genVec, int NumCount, const char* fileName, bool defaultVec) {
			// Input: module, initial value of x
			int p = 624;
			int w = 32;
			//string uBin = "10000000000000000000000000000000";
			//string hBin = "01111111111111111111111111111111";
			long long uDec = 2147483647;
			long long hDec = 2147483648;
			long long r = 31;
			long long q = 397;
			long long a = 2567483615;
			long long u = 11;
			long long s = 7;
			long long t = 15;
			long long l = 18;
			long long b = 2636928640;
			long long c = 4022730752;
			if (defaultVec) {
				genVec = { 10001, 8191 };
			}
			long long zMod = genVec[0];
			long long x0 = genVec[1];
			vector <long long> x_zns(1, x0);
			for (int i = 0; i < p; i++)
			{
				x_zns.push_back(abs(1812433253 * (x_zns[i - 1] ^ (x_zns[i - 1] >> 30)) + i));
			}
			
			vector<int> showsVector = progressVector(NumCount);
			int check_progress_index = 0;
			
			ofstream f;
			f.open(fileName, ios::out);
			int n = 0;
			for (int countN = 0; countN < NumCount; countN++)
			{
				long long xn = x_zns[n];
				long long xnPlus1 = x_zns[(n + 1) % p];
				long long yNew = (xn & uDec) | (xnPlus1 & hDec);
				int yNewLength = log2(yNew) + 1;
				deque<bool> yBin = binary(yNew, yNewLength);
				long long Xnew;
				yNew >>= 1;
				long long xnqp = x_zns[(n + q) % p];
				if (yBin[yBin.size() - 1] == 1)
				Xnew = xnqp ^ yNew ^ a;
				else
				Xnew = xnqp ^ yNew ^ 0;
				yNew = Xnew;
				yNew = yNew ^ (yNew >> u);
				yNew = yNew ^ ((yNew << s) & b);
				yNew = yNew ^ ((yNew << t) & c);
				long long zNew = yNew ^ (yNew >> l);
				if (defaultVec)
				f << (zNew % zMod) % 1001 << ",";
				else
				f << zNew % zMod << ",";
				
				if (countN == showsVector[check_progress_index]) {
					cout << (check_progress_index + 1) * 10 << "% completed\n";
					check_progress_index++;
				}
				x_zns[n] = Xnew;
				n = (n + 1) % p;
			}
			f.close();
			return;
		}
	\end{minted}
	
	
	\subsection{RC4}
	
	Описание алгоритма:
	
	\begin{enumerate}
		\item Инициализация $S_i, i = 0,1, \dots ,255.$
		\begin{enumerate}
			\item  for $i$ = 0 to 255: $ S_i$ = $i $
			\item $ j$ = 0;
			\item  for $i$ = 0 to 255: $j$ = $(j + S_i + K_i)$ mod 256; $Swap(S_i, S_j) $
		\end{enumerate}
		\item $i = 0, j = 0$.
		\item Итерация алгоритма:
		\begin{enumerate}
			\item $i = (i + 1) \mod 256$;
			\item $j = (j + S_i) \mod 256$;
			\item $Swap(S_i, S_j)$;
			\item $t = (S_i + S_j) \mod 256$;
			\item $K = S_t$.
		\end{enumerate}
	\end{enumerate} \
	
	Инициализирующий вектор: 256 начальных значений.
	
	Параметры запуска программы:
	
	\textit{prng.exe /n:16000 /g:rc4 /i:802,720,341,337,961,882,417,785,198,727,899,372, 374,425,556,615,813,768,840,183,893,568,73,387,18,436,182,125,806,899,485,607, 619,825,944,579,707,360,363,904,87,262,276,460,687,831,75,499,599,915,681,492, 483,754,878,500,189,60,624,994,959,109,600,577,934,544,156,640,903,519,544,990, 781,819,449,468,650,524,967,248,438,647,739,920,400,617,419,588,676,43,581,634, 151,181,211,84,724,367,723,627,886,267,617,667,85,65,134,735,589,100,983,26,747, 721,945,147,337,364,734,13,406,315,647,556,496,858,640,220,224,362,847,110,629, 463,776,713,528,909,448,116,9,430,141,755,151,86,901,488,449,635,500,855,950,147, 410,446,4,49,665,227,411,511,336,39,974,112,752,501,21,200,617,29,629,757,784,779, 843,684,266,292,319,766,146,269,912,556,714,916,605,378,142,15,889,478,54,862,590, 806,363,610,5,979,638,634,736,421,413,578,105,679,869,424,444,14,692,356,569,405, 271,173,783,413,188,671,891,242,533,480,48,895,89,53,873,727,686,608,147,98, 185,} \textit{252,776,54,675,220,67,366,576,636,771,846,808,553,259,996,224,149 /f:prngTestRc4.txt}
	
	\begin{figure}[H]
		\centering
		\includegraphics[width=0.9\textwidth]{rc4Cmd}
		\caption{Запуск программы}
		\label{fig:rc4Cmd}
	\end{figure}
	
	\begin{figure}[H]
		\centering
		\includegraphics[width=0.9\textwidth]{rc4Test}
		\caption{Сгенерированная последовательность}
		\label{fig:rc4Test}
	\end{figure}	
	
		Исходный код программы:
	
	\begin{minted}[fontsize=\footnotesize]{cpp}
	void rc4Func(string codeMethod, vector <int> genVec, int NumCount, const char* fileName, bool defaultVec) {
		// Input: 256 initial values
		if (defaultVec) {
			genVec = { 802, 720, 341, 337, 961, 882, 417, 785, 198, 727, 899, 372, 374, 425, 556, 615, 813, 768, 840, 183, 893, 568, 73, 387, 18, 436, 182, 125, 806, 899, 485, 607, 619, 825, 944, 579, 707, 360, 363, 904, 87, 262, 276, 460, 687, 831, 75, 499, 599, 915, 681, 492, 483, 754, 878, 500, 189, 60, 624, 994, 959, 109, 600, 577, 934, 544, 156, 640, 903, 519, 544, 990, 781, 819, 449, 468, 650, 524, 967, 248, 438, 647, 739, 920, 400, 617, 419, 588, 676, 43, 581, 634, 151, 181, 211, 84, 724, 367, 723, 627, 886, 267, 617, 667, 85, 65, 134, 735, 589, 100, 983, 26, 747, 721, 945, 147, 337, 364, 734, 13, 406, 315, 647, 556, 496, 858, 640, 220, 224, 362, 847, 110, 629, 463, 776, 713, 528, 909, 448, 116, 9, 430, 141, 755, 151, 86, 901, 488, 449, 635, 500, 855, 950, 147, 410, 446, 4, 49, 665, 227, 411, 511, 336, 39, 974, 112, 752, 501, 21, 200, 617, 29, 629, 757, 784, 779, 843, 684, 266, 292, 319, 766, 146, 269, 912, 556, 714, 916, 605, 378, 142, 15, 889, 478, 54, 862, 590, 806, 363, 610, 5, 979, 638, 634, 736, 421, 413, 578, 105, 679, 869, 424, 444, 14, 692, 356, 569, 405, 271, 173, 783, 413, 188, 671, 891, 242, 533, 480, 48, 895, 89, 53, 873, 727, 686, 608, 147, 98, 185, 252, 776, 54, 675, 220, 67, 366, 576, 636, 771, 846, 808, 553, 259, 996, 224, 149 };
		}
		vector <int> k = genVec;
		vector <int> s = genVec;
		int j = 0;
		for (int i = 0; i < 256; i++)
		{
			j = (j + s[i] + k[i]) % 256;
			swap(s[i], s[j]);
		}
		int i = 0;
		j = 0;
		
		vector<int> showsVector = progressVector(NumCount);
		int check_progress_index = 0;
		
		ofstream f;
		f.open(fileName, ios::out);
		for (int countN = 0; countN < NumCount; countN++)
		{
			i = (i + 1) % 256;
			j = (j + s[j]) % 256;
			swap(s[i], s[j]);
			int t = (s[i] + s[j]) % 256;
			if (defaultVec)
			f << s[t] % 1001 << ",";
			else
			f << s[t] << ",";
			
			if (countN == showsVector[check_progress_index]) {
				cout << (check_progress_index + 1) * 10 << "% completed\n";
				check_progress_index++;
			}
		}
		f.close();
		return;
	}
	\end{minted}
	
	
	\subsection{ГПСЧ на основе RSA}
	
	Описание алгоритма:
	
	\begin{enumerate}
		\item Инициализация простых чисел $p$, $q$ и $n = pq$, а также $f = (p - 1)(q - 1)$. Выбрать случайное число $e$: $1 < e < f$, НОД($e, f$) = 1.
		\item Выбрать случайное целое $x_0$ из интервала $[1, n - 1]$.
		\item For $i$ = 1 to $w$ do
		\begin{enumerate}
			\item $x_i \leftarrow x_{i-1}^e \mod n$.
			\item $z_i \leftarrow $ последний значащий бит $x_i$
		\end{enumerate}
		\item Вернуть $z_1, \dots, z_w$.
	\end{enumerate} \
	
	Инициализирующий вектор: модуль $n$, число $e$, начальное значение $x$.
	
	Параметры запуска программы:
	
	\textit{prng.exe /n:16000 /g:rsa /i:7191817,151,69 /f:prngTestRsa.txt}
	
	\begin{figure}[H]
		\centering
		\includegraphics[width=0.9\textwidth]{rsaCmd}
		\caption{Запуск программы}
		\label{fig:rsaCmd}
	\end{figure}
	
	\begin{figure}[H]
		\centering
		\includegraphics[width=0.9\textwidth]{rsaTest}
		\caption{Сгенерированная последовательность}
		\label{fig:rsaTest}
	\end{figure}

		Исходный код программы:
	
	\begin{minted}[fontsize=\footnotesize]{cpp}
		void rsaFunc(string codeMethod, vector <int> genVec, int NumCount, const char* fileName, bool defaultVec) {
			// Input: module n, number e, initial value of x; e satisfies conditions: 1 < e < (p-1)(q-1), NOD(e, (p-1)(q-1)) = 1, where p*q=n. x from the interval [1,n]
			if (defaultVec) {
				genVec = { 7191817, 151, 69 };
			}
			int n = genVec[0];
			int e = genVec[1];
			int x = genVec[2];
			int l = 20;
			
			if (e <= 1 || x < 1 || x > n - 1) {
				cout << "\nError. The parameters do not meet the requirements.\n";
				return;
			}
			
			vector<int> showsVector = progressVector(NumCount);
			int check_progress_index = 0;
			
			ofstream f;
			f.open(fileName, ios::out);
			for (int countN = 0; countN < NumCount; countN++)
			{
				string zs = "";
				for (int i = 0; i < l; i++)
				{
					x = powmod(x, e, n);
					char z = '0';
					if (x % 2 != 0)
					z = '1';
					zs.push_back(z);
				}
				if (defaultVec)
				f << binToDec(zs) % 1001 << ",";
				else
				f << binToDec(zs) << ",";
				
				if (countN == showsVector[check_progress_index]) {
					cout << (check_progress_index + 1) * 10 << "% completed\n";
					check_progress_index++;
				}
			}
			f.close();
			return;
		}
	\end{minted}
	

	\subsection{Алгоритм Блюм-Блюма-Шуба}
	
	Описание алгоритма:
	
	\textbf{На входе}: Длина $l$.
	
	\textbf{На выходе}: Последовательность псевдослучайных бит $z_1, z_2, \dots, z_l$.

	\begin{enumerate}
		\item Сгенерировать два простых числа $p$ и $q$, сравнимых с 3 по модулю 4. Произведение этих чисел, $n$, является целым числом Блюма. Выберем другое случайное целое число $x$, взаимно простое с $n$.
		\item Вычислим $x_0 = x^2 \mod n$, которое будет начальным вектором.
		\item For $i$ = 1 to $w$ do
		\begin{enumerate}
			\item $x_i \leftarrow x_{i-1}^2 \mod n$.
			\item $z_i \leftarrow $ последний значащий бит $x_i$
		\end{enumerate}
		\item Вернуть $z_1, \dots, z_w$.
	\end{enumerate} \
	
	Инициализирующий вектор: начальное значение $x$ (взаимно простое с $n$). 
	
	Параметры запуска программы:
	
	\textit{prng.exe /n:16000 /g:bbs /i:8627 /f:prngTestBbs.txt}
	
	\begin{figure}[H]
		\centering
		\includegraphics[width=0.9\textwidth]{bbsCmd}
		\caption{Запуск программы}
		\label{fig:bbsCmd}
	\end{figure}
	
	\begin{figure}[H]
		\centering
		\includegraphics[width=0.9\textwidth]{bbsTest}
		\caption{Сгенерированная последовательность}
		\label{fig:bbsTest}
	\end{figure}

		Исходный код программы:
	
	\begin{minted}[fontsize=\footnotesize]{cpp}
		void bbsFunc(string codeMethod, vector <int> genVec, int NumCount, const char* fileName, bool defaultVec) {
			// Input: Initial value of x (mutually prime with n)
			if (defaultVec) {
				genVec = { 8627 };
			}
			int x0 = genVec[0];
			int n = 16637;
			int l = 10;
			
			if (nod(x0, n) != 1) {
				cout << "\nThe parameters do not meet the requirements (x and n are not mutually prime).\n";
			}
			
			vector<int> showsVector = progressVector(NumCount);
			int check_progress_index = 0;
			
			ofstream f;
			f.open(fileName, ios::out);
			for (int countN = 0; countN < NumCount; countN++)
			{
				x0 = (x0 * x0) % n;
				string zs = "";
				for (int i = 0; i < l; i++)
				{
					x0 = (x0 * x0) % n;
					char z = '0';
					if (x0 % 2 != 0)
					z = '1';
					zs.push_back(z);
				}
				if (defaultVec)
				f << binToDec(zs) % 1001 << ",";
				else
				f << binToDec(zs) << ",";
				
				if (countN == showsVector[check_progress_index]) {
					cout << (check_progress_index + 1) * 10 << "% completed\n";
					check_progress_index++;
				}
			}
			f.close();
			return;
		}
	\end{minted}

	\section{Преобразование ПСЧ к заданному распределению}

	Описание задания:

	Создать программу для преобразования последовательности ПСЧ в другую последовательность
	ПСЧ с заданным распределением:
	
	\begin{enumerate}
		\item Стандартное равномерное с заданным интервалом;
		\item Треугольное распределение;
		\item Общее экспоненциальное распределение;
		\item Нормальное распределение;
		\item Гамма распределение (для параметра c=k);
		\item Логнормальное распределение;
		\item Логистическое распределение;
		\item Биномиальное распределение.
	\end{enumerate}

	Для преобразования последовательности ПСЧ в другую последовательность
	ПСЧ с заданным распределением использовалась последовательность, полученная с помощью вихря Мерсенна:
	
	\begin{figure}[H]
		\centering
		\includegraphics[width=0.9\textwidth]{mtTest2}
		\caption{Последовательность, полученная с помощью вихря Мерсенна}
		\label{fig:mtTest2}
	\end{figure}
	
	Если максимальное значение равномерного целого случайного числа $X$
	равно $(m - 1)$, для генерации стандартных равномерных случайных чисел 
	необходимо применять следующую формулу: $U = X / m$.
	
	\subsection{Стандартное равномерное с заданным интервалом}
	
	Описание алгоритма:
	
	Если число $U$ соответствует стандартному равномерному распределению, то равномерное случайное число должно быть получено в соответствии со следующей формулой:
 
	\begin{center}
		$Y = bU + a$
	\end{center}
	
	Параметры запуска программы:
	
	\textit{rnc.exe /d:st /f:prngTestMt.txt /p1:75 /p2:200}

	
	\begin{figure}[H]
		\centering
		\includegraphics[width=0.9\textwidth]{stCmdRnc}
		\caption{Запуск программы}
		\label{fig:stCmdRnc}
	\end{figure}
	
	\begin{figure}[H]
		\centering
		\includegraphics[width=0.9\textwidth]{stTestRnc}
		\caption{Последовательность с заданным распределением}
		\label{fig:stTestRnc}
	\end{figure}

	\begin{figure}[H]
		\centering
		\includegraphics[width=0.7\textwidth]{stPy}
		\caption{График распределения}
		\label{fig:stPy}
	\end{figure}
	
	Исходный код программы:
	
	\begin{minted}[fontsize=\footnotesize]{cpp}
	vector <double> stFunc(string distributionMethod, string inputFileName, string parameter1, string parameter2, bool defaultVec, bool& errorMessage) {
		ifstream inFile(inputFileName);
		string lineFile;
		vector <double> allElems;
		if (!inFile.is_open()) {
			cout << "Ошибка открытия файла " << inputFileName << endl;
			errorMessage = true;
			return allElems;
		}
		string getlineElem;
		getline(inFile, getlineElem);
		stringstream ss(getlineElem);
		while (getline(ss, getlineElem, ',')) {
			double intElem = stod(getlineElem);
			allElems.push_back(intElem);
		}
		inFile.close();
		
		vector <double> allU = XtoU(allElems);
		double a = stod(parameter1);
		double b = stod(parameter2);
		for (int i = 0; i < allU.size(); i++)
		{
			allU[i] = (b * allU[i]) + a;
		}
		
		return allU;
	}
	\end{minted}

	\subsection{Треугольное распределение}
	
	Описание алгоритма:
	
	Если стандартные случайные числа $U_1$ и $U_2$ независимо получены методом генерации стандартного равномерного числа, то случайное число $Y$, подчиняющееся треугольному распределению, определяют по формуле:
	
	\begin{center}
		$Y = a + b(U_1 + U_2 - 1)$
	\end{center}

	Параметры запуска программы:
	
	\textit{rnc.exe /d:tr /f:prngTestMt.txt /p1:1 /p2:1000}
	
	\begin{figure}[H]
		\centering
		\includegraphics[width=0.9\textwidth]{trCmdRnc}
		\caption{Запуск программы}
		\label{fig:trCmdRnc}
	\end{figure}
	
	\begin{figure}[H]
		\centering
		\includegraphics[width=0.9\textwidth]{trTestRnc}
		\caption{Последовательность с заданным распределением}
		\label{fig:trTestRnc}
	\end{figure}
	
	\begin{figure}[H]
		\centering
		\includegraphics[width=0.7\textwidth]{trPy}
		\caption{График распределения}
		\label{fig:trPy}
	\end{figure}
	
	Исходный код программы:
	
	\begin{minted}[fontsize=\footnotesize]{cpp}
		vector <double> trFunc(string distributionMethod, string inputFileName, string parameter1, string parameter2, bool defaultVec, bool& errorMessage) {
			ifstream inFile(inputFileName);
			string lineFile;
			vector <double> allElems;
			if (!inFile.is_open()) {
				cout << "Ошибка открытия файла " << inputFileName << endl;
				errorMessage = true;
				return allElems;
			}
			string getlineElem;
			getline(inFile, getlineElem);
			stringstream ss(getlineElem);
			while (getline(ss, getlineElem, ',')) {
				double intElem = stod(getlineElem);
				allElems.push_back(intElem);
			}
			inFile.close();
			
			vector <double> allU = XtoU(allElems);
			double a = stod(parameter1);
			double b = stod(parameter2);
			for (int i = 0; i < allU.size() - 1; i++)
			{
				allU[i] = a + (b * (allU[i] + allU[i + 1] - 1));
			}
			
			return allU;
		}
	\end{minted}
	

	\subsection{Общее экспоненциальное распределение}
	
	Описание алгоритма:
	
	Если стандартное равномерное случайное число $U$ генерировано одним из методов, установленным в разделе 2, то случайное число, соответствующее экспоненциальному распределению, получают по формуле:
	
	\begin{center}
		$Y = -b \ln(U) + a$
	\end{center}
	
	Параметры запуска программы:
	
	\textit{rnc.exe /d:ex /f:D:prngTestMt.txt /p1:7 /p2:12}
	
	\begin{figure}[H]
		\centering
		\includegraphics[width=0.9\textwidth]{exCmdRnc}
		\caption{Запуск программы}
		\label{fig:exCmdRnc}
	\end{figure}
	
	\begin{figure}[H]
		\centering
		\includegraphics[width=0.9\textwidth]{exTestRnc}
		\caption{Последовательность с заданным распределением}
		\label{fig:exTestRnc}
	\end{figure}
	
	\begin{figure}[H]
		\centering
		\includegraphics[width=0.7\textwidth]{exPy}
		\caption{График распределения}
		\label{fig:exPy}
	\end{figure}
	
	Исходный код программы:
	
	\begin{minted}[fontsize=\footnotesize]{cpp}
		vector <double> exFunc(string distributionMethod, string inputFileName, string parameter1, string parameter2, bool defaultVec, bool& errorMessage) {
			ifstream inFile(inputFileName);
			string lineFile;
			vector <double> allElems;
			if (!inFile.is_open()) {
				cout << "Ошибка открытия файла " << inputFileName << endl;
				errorMessage = true;
				return allElems;
			}
			string getlineElem;
			getline(inFile, getlineElem);
			stringstream ss(getlineElem);
			while (getline(ss, getlineElem, ',')) {
				double intElem = stod(getlineElem);
				allElems.push_back(intElem);
			}
			inFile.close();
			
			vector <double> allU = XtoU(allElems);
			double a = stod(parameter1);
			double b = stod(parameter2);
			for (int i = 0; i < allU.size() - 1; i++)
			{
				if (allU[i] == 0) {
					cout << "Одно из случайных чисел U = 0. Ошибка в вычислении ln(U)" << endl;
					errorMessage = true;
					return allU;
				}
				allU[i] = ((b * (-1)) * log(allU[i])) + a;
			}
			
			return allU;
		}
	\end{minted}

	
	\subsection{Нормальное распределение}

	Описание алгоритма:

	Если стандартные равномерные случайные числа $U_1$ и $U_2$ независимо сгенерированы методом, установленным в разделе 2, то два независимых нормальных случайных числа $Z_1$,$Z_2$ получают в соответствии со следующей процедурой
	
	\begin{center}
		$Z_1 = \mu + \sigma \sqrt{-2 \ln(1 - U_1)} \cos(2 \pi U_2)$
	\end{center}

	\begin{center}
		$Z_2 = \mu + \sigma \sqrt{-2 \ln(1 - U_1)} \sin(2 \pi U_2)$
	\end{center}
	
	Параметры запуска программы:
	
	\textit{rnc.exe /d:nr /f:D:prngTestMt.txt /p1:1 /p2:1000}
	
	
	\begin{figure}[H]
		\centering
		\includegraphics[width=0.9\textwidth]{nrCmdRnc}
		\caption{Запуск программы}
		\label{fig:nrCmdRnc}
	\end{figure}
	
	\begin{figure}[H]
		\centering
		\includegraphics[width=0.9\textwidth]{nrTestRnc}
		\caption{Последовательность с заданным распределением}
		\label{fig:nrTestRnc}
	\end{figure}
	
	\begin{figure}[H]
		\centering
		\includegraphics[width=0.7\textwidth]{nrPy}
		\caption{График распределения}
		\label{fig:nrPy}
	\end{figure}
	
	Исходный код программы:
	
	\begin{minted}[fontsize=\footnotesize]{cpp}
		vector <double> nrFunc(string distributionMethod, string inputFileName, string parameter1, string parameter2, bool defaultVec, bool& errorMessage) {
			ifstream inFile(inputFileName);
			string lineFile;
			vector <double> allElems;
			if (!inFile.is_open()) {
				cout << "Ошибка открытия файла " << inputFileName << endl;
				errorMessage = true;
				return allElems;
			}
			string getlineElem;
			getline(inFile, getlineElem);
			stringstream ss(getlineElem);
			while (getline(ss, getlineElem, ',')) {
				double intElem = stod(getlineElem);
				allElems.push_back(intElem);
			}
			inFile.close();
			
			if (allElems.size() % 2 != 0)
			allElems.erase(allElems.end() - 1);
			
			vector <double> allU = XtoU(allElems);
			double mu = stod(parameter1);
			double sigma = stod(parameter2);
			for (int i = 0; i < allU.size(); i += 2)
			{
				if (allU[i] == 1) {
					cout << "Одно из случайных чисел U = 1. Ошибка в вычислении ln(1 - U)" << endl;
					errorMessage = true;
					return allU;
				}
				double U1 = allU[i];
				double U2 = allU[i + 1];
				allU[i] = mu + (sigma * sqrt(-2 * log(1 - U1)) * cos(2 * M_PI * U2)); // Z1
				allU[i + 1] = mu + (sigma * sqrt(-2 * log(1 - U1)) * sin(2 * M_PI * U2)); // Z2
			}
			
			return allU;
		}
	\end{minted}

	
	\subsection{Гамма распределение}
	
	Описание алгоритма:

	Используя независимую равномерную случайную величину $U_1$, т.к. рассматривается случай $(c = 1)$, применяем формулу
	
	\begin{center}
		$Y = a - b \ln(1 - U_1)$
	\end{center}
	
	Параметры запуска программы:
	
	\textit{rnc.exe /d:gm /f:D:prngTestMt.txt /p1:7 /p2:12}
	
	
	\begin{figure}[H]
		\centering
		\includegraphics[width=0.9\textwidth]{gmCmdRnc}
		\caption{Запуск программы}
		\label{fig:gmCmdRnc
	\end{figure}
	
	\begin{figure}[H]
		\centering
		\includegraphics[width=0.9\textwidth]{gmTestRnc}
		\caption{Последовательность с заданным распределением}
		\label{fig:gmTestRnc}
	\end{figure}
	
	\begin{figure}[H]
		\centering
		\includegraphics[width=0.7\textwidth]{gmPy}
		\caption{График распределения}
		\label{fig:gmPy}
	\end{figure}
	
	Исходный код программы:
	
	\begin{minted}[fontsize=\footnotesize]{cpp}
		vector <double> gmFunc(string distributionMethod, string inputFileName, string parameter1, string parameter2, bool defaultVec, bool& errorMessage) {
			ifstream inFile(inputFileName);
			string lineFile;
			vector <double> allElems;
			if (!inFile.is_open()) {
				cout << "Ошибка открытия файла " << inputFileName << endl;
				errorMessage = true;
				return allElems;
			}
			string getlineElem;
			getline(inFile, getlineElem);
			stringstream ss(getlineElem);
			while (getline(ss, getlineElem, ',')) {
				double intElem = stod(getlineElem);
				allElems.push_back(intElem);
			}
			inFile.close();
			
			vector <double> allU = XtoU(allElems);
			double a = stod(parameter1);
			double b = stod(parameter2);
			for (int i = 0; i < allU.size(); i++)
			{
				if (allU[i] == 1) {
					cout << "Одно из случайных чисел U = 1. Ошибка в вычислении ln(1 - U)" << endl;
					errorMessage = true;
					return allU;
				}
				allU[i] = a - (b * log(1 - allU[i]));
			}
			
			return allU;
		}
	\end{minted}

	
	\subsection{Логнормальное распределение}

	Описание алгоритма:

	Используя стандартные нормальные случайные числа $Z$, применяют формулу
	
	\begin{center}
		$ Y = a + \exp(b - Z) $
	\end{center}
	
	для получения случайных чисел, соответствующих логнормальному распределению. \
	
	Параметры запуска программы:
	
	\textit{rnc.exe /d:ln /f:D:prngTestMt.txt /p1:2 /p2:1,1}
	
	
	\begin{figure}[H]
		\centering
		\includegraphics[width=0.9\textwidth]{lnCmdRnc}
		\caption{Запуск программы}
		\label{fig:lnCmdRnc}
	\end{figure}
	
	\begin{figure}[H]
		\centering
		\includegraphics[width=0.9\textwidth]{lnTestRnc}
		\caption{Последовательность с заданным распределением}
		\label{fig:lnTestRnc}
	\end{figure}
	
	\begin{figure}[H]
		\centering
		\includegraphics[width=0.7\textwidth]{lnPy}
		\caption{График распределения}
		\label{fig:lnPy}
	\end{figure}
	
	Исходный код программы:
	
	\begin{minted}[fontsize=\footnotesize]{cpp}
		vector <double> lnFunc(string distributionMethod, string inputFileName, string parameter1, string parameter2, bool defaultVec, bool& errorMessage) {
			ifstream inFile(inputFileName);
			string lineFile;
			vector <double> allElems;
			if (!inFile.is_open()) {
				cout << "Ошибка открытия файла " << inputFileName << endl;
				errorMessage = true;
				return allElems;
			}
			string getlineElem;
			getline(inFile, getlineElem);
			stringstream ss(getlineElem);
			while (getline(ss, getlineElem, ',')) {
				double intElem = stod(getlineElem);
				allElems.push_back(intElem);
			}
			inFile.close();
			
			if (allElems.size() % 2 != 0)
			allElems.erase(allElems.end() - 1);
			
			vector <double> allU = XtoU(allElems);
			double a = stod(parameter1);
			double b = stod(parameter2);
			for (int i = 0; i < allU.size(); i += 2)
			{
				if (allU[i] == 1) {
					cout << "Одно из случайных чисел U = 1. Ошибка в вычислении ln(1 - U)" << endl;
					errorMessage = true;
					return allU;
				}
				double U1 = allU[i];
				double U2 = allU[i + 1];
				double Z1 = a + (b * sqrt(-2 * log(1 - U1)) * cos(2 * M_PI * U2));
				double Z2 = a + (b * sqrt(-2 * log(1 - U1)) * sin(2 * M_PI * U2));
				double bminZ1 = b - Z1;
				if (bminZ1 > 43) {
					cout << "Разность b - Z1 больше 43. Выход за границы double в вычислении exp(b - Z1)" << endl;
					errorMessage = true;
					return allU;
				}
				double bminZ2 = b - Z2;
				if (bminZ1 > 43) {
					cout << "Разность b - Z2 больше 43. Выход за границы double в вычислении exp(b - Z2)" << endl;
					errorMessage = true;
					return allU;
				}
				allU[i] = a + exp(bminZ1);
				allU[i + 1] = a + exp(bminZ2);
			}
			
			return allU;
		}
	\end{minted}

	
	\subsection{Логистическое распределение}

	Описание алгоритма:

	Если стандартные равномерные случайные числа $U$ генерированы методом, изложенным выше, то случайные числа, соответствующие логистическому распределению, получают по формуле:
	
	\begin{center}
		$Y = a + b \ln(\frac{U}{1 - U})$
	\end{center}
	
	Параметры запуска программы:
	
	\textit{rnc.exe /d:ls /f:D:prngTestMt.txt /p1:7 /p2:16}
	
	
	\begin{figure}[H]
		\centering
		\includegraphics[width=0.9\textwidth]{lsCmdRnc}
		\caption{Запуск программы}
		\label{fig:lsCmdRnc}
	\end{figure}
	
	\begin{figure}[H]
		\centering
		\includegraphics[width=0.9\textwidth]{lsTestRnc}
		\caption{Последовательность с заданным распределением}
		\label{fig:lsTestRnc}
	\end{figure}
	
	\begin{figure}[H]
		\centering
		\includegraphics[width=0.7\textwidth]{lsPy}
		\caption{График распределения}
		\label{fig:lsPy}
	\end{figure}
	
	Исходный код программы:
	
	\begin{minted}[fontsize=\footnotesize]{cpp}
	vector <double> lsFunc(string distributionMethod, string inputFileName, string parameter1, string parameter2, bool defaultVec, bool& errorMessage) {
		ifstream inFile(inputFileName);
		string lineFile;
		vector <double> allElems;
		if (!inFile.is_open()) {
			cout << "Ошибка открытия файла " << inputFileName << endl;
			errorMessage = true;
			return allElems;
		}
		string getlineElem;
		getline(inFile, getlineElem);
		stringstream ss(getlineElem);
		while (getline(ss, getlineElem, ',')) {
			double intElem = stod(getlineElem);
			allElems.push_back(intElem);
		}
		inFile.close();
		
		vector <double> allU = XtoU(allElems);
		double a = stod(parameter1);
		double b = stod(parameter2);
		for (int i = 0; i < allU.size(); i += 2)
		{
			if (allU[i] == 1) {
				cout << "Одно из случайных чисел U = 1. Ошибка в вычислении U/(1 - U). Деление на 0." << endl;
				errorMessage = true;
				return allU;
			}
			else if (allU[i] == 0) {
				cout << "Одно из случайных чисел U = 0. Ошибка в вычислении ln(U/(1 - U))" << endl;
				errorMessage = true;
				return allU;
			}
			allU[i] = a + (b * log(allU[i] / (1 - allU[i])));
		}
		
		return allU;
	}	
	\end{minted}

	
	\subsection{Биномиальное распределение}

	Описание алгоритма:

	Вычисляют функцию распределения
	
	\begin{center}
	$F(y) = \sum_{k=0}^{y} C_n^k  p^k (1-p)^{n-k},$ $ y = 0, 1, \dots , n$
	\end{center}
	
	Для получения случайного числа $Y$ генерируют стандартное равномерное случайное число $U$. Случайное число $Y$ является наименьшим значением $y$, для которого $U \leq F(y)$. \
	
	Параметры запуска программы:
	
	\textit{rnc.exe /d:bi /f:D:prngTestMt.txt /p1:10 /p2:0,5}
	
	
	\begin{figure}[H]
		\centering
		\includegraphics[width=0.9\textwidth]{biCmdRnc}
		\caption{Запуск программы}
		\label{fig:biCmdRnc}
	\end{figure}
	
	\begin{figure}[H]
		\centering
		\includegraphics[width=0.9\textwidth]{biTestRnc}
		\caption{Последовательность с заданным распределением}
		\label{fig:biTestRnc}
	\end{figure}
	
	\begin{figure}[H]
		\centering
		\includegraphics[width=0.7\textwidth]{biPy}
		\caption{График распределения}
		\label{fig:biPy}
	\end{figure}
	
	Исходный код программы:
	
	\begin{minted}[fontsize=\footnotesize]{cpp}
		double Cnk(double n, double k) {
			double nfact = 1;
			for (int i = 1; i <= n; i++) {
				nfact = nfact * i;
			}
			double nminkfact = 1;
			for (int i = 1; i <= n - k; i++) {
				nminkfact = nminkfact * i;
			}
			double kfact = 1;
			for (int i = 1; i <= k; i++) {
				kfact = kfact * i;
			}
			double res = nfact / (nminkfact * kfact);
			return res;
		}
		
		double funcBiF(double y, double n, double p) {
			double resSum = 0;
			for (int k = 0; k <= y; k++)
			{
				resSum = resSum + (Cnk(n, k) * pow(p, k) * pow(1 - p, n - k));
			}
			return resSum;
		}
		
		vector <double> biFunc(string distributionMethod, string inputFileName, string parameter1, string parameter2, bool defaultVec, bool& errorMessage) {
			ifstream inFile(inputFileName);
			string lineFile;
			vector <double> allElems;
			if (!inFile.is_open()) {
				cout << "Ошибка открытия файла " << inputFileName << endl;
				errorMessage = true;
				return allElems;
			}
			string getlineElem;
			getline(inFile, getlineElem);
			stringstream ss(getlineElem);
			while (getline(ss, getlineElem, ',')) {
				double intElem = stod(getlineElem);
				allElems.push_back(intElem);
			}
			inFile.close();
			
			vector <double> allU = XtoU(allElems);
			double n = stod(parameter1); //n
			double p = stod(parameter2); //p
			if (p > 1 || p < 0) {
				cout << "Ошибка. Параметр p2 не может быть > 1 или < 0, т.к. это вероятность." << endl;
				errorMessage = true;
				return allElems;
			}
			for (int i = 0; i < allU.size(); i++)
			{
				double y = 0;
				for (int j = 0; j <= n; j++)
				{
					if (allU[i] > funcBiF(y, n, p)) {
						y++;
						if (j == n) {
							cout << "Ошибка. Не нашлось такого y = (0,...,n), при котором U <= F(y)" << endl;
							errorMessage = true;
							return allElems;
						}
					}
					else
					break;
				}
				allU[i] = y;
			}
			
			return allU;
		}
	\end{minted}


	\newpage
	\appendix
	\section{Исходный код задания 1}
	
	\begin{minted}[fontsize=\footnotesize]{cpp}
		#include <iostream>
		#include <string>
		#include <vector>
		#include <fstream>
		#include <iomanip>
		#include <bitset>
		#include <sstream>
		#include <deque>
		#include <cmath>
		
		using namespace std;
		
		deque<bool> binary(int x0, int num_bits) {
			bitset<sizeof(int) * 8> bits(x0);
			deque<bool> binary_digits;
			for (int i = num_bits - 1; i >= 0; --i) {
				binary_digits.push_back(bits[i]);
			}
			return binary_digits;
		}
		
		long long binToDec(string binStr)
		{
			short i;
			long long res = 0;
			for (i = 0; i < sizeof(long long) * 8 && binStr[i]; ++i)
			{
				res <<= 1;
				res |= (binStr[i] - '0');
			}
			return res;
		}
		
		long long binDequeToDec(deque<bool> binStr)
		{
			short i;
			long long res = 0;
			for (i = 0; i < sizeof(long long) * 8 && binStr[i]; ++i)
			{
				res <<= 1;
				res |= (binStr[i] - '0');
			}
			return res;
		}
		
		vector<int> splitStr(string str, char symbol)
		{
			vector<int> elems;
			int elem;
			stringstream ss(str);
			while (ss >> elem) {
				elems.push_back(elem);
				if (ss.peek() == symbol) {
					ss.ignore();
				}
			}
			return elems;
		}
		
		pair<vector<int>, vector <deque <bool>>> splitStrFsr(string str, char symbol, string codeMethod)
		{
			vector<int> elems;
			vector <deque <bool>> nfsrRegs;
			if (codeMethod == "lfsr") {
				char elem;
				int k = 0;
				vector<string> elemsLfsr(2);
				stringstream ss(str);
				while (ss >> elem) {
					elemsLfsr[k].push_back(elem);
					if (ss.peek() == symbol) {
						ss.ignore();
						k++;
					}
				}
				elems.push_back(stoi(elemsLfsr[1]));
				int elemsLfsrSize = elemsLfsr[0].size();
				for (int i = 0; i < elemsLfsrSize; i++)
				{
					if (elemsLfsr[0][i] == '1')
					{
						elems.push_back(elemsLfsrSize - 1 - i);
					}
				}
				//x0,coeffs
			}
			else {
				char elem;
				int k = 0;
				vector<string> elemsNfsr(3);
				stringstream ss(str);
				while (ss >> elem) {
					elemsNfsr[k].push_back(elem);
					if (ss.peek() == symbol) {
						ss.ignore();
						k++;
					}
				}
				for (int j = 0; j < elemsNfsr.size(); j++)
				{
					int elemsNfsrSize = elemsNfsr[j].size();
					int coeffsCount = 0;
					vector <int> elemsDop;
					for (int i = 0; i < elemsNfsrSize; i++)
					{
						if (elemsNfsr[j][i] == '1')
						{
							elemsDop.push_back(elemsNfsrSize - 1 - i);
							coeffsCount++;
						}
					}
					elems.push_back(coeffsCount);
					for (int i = 0; i < coeffsCount; i++)
					{
						elems.push_back(elemsDop[i]);
					}
				}
				//size1,coeffs1,size2,coeffs2,size3,coeffs3
				for (int i = 0; i < 3; i++)
				{
					deque<bool> strInDeq;
					for (int j = 0; j < elemsNfsr[i].size(); j++)
					{
						if (elemsNfsr[i][j] == '1')
						strInDeq.push_back(1);
						else
						strInDeq.push_back(0);
					}
					nfsrRegs.push_back(strInDeq);
				}
			}
			pair<vector<int>, vector <deque <bool>>> resPair = { elems , nfsrRegs };
			return resPair;
		}
		
		deque<bool> multRs(deque<bool> r1, deque<bool> r2) {
			int r1size = r1.size();
			int r2size = r2.size();
			if (r1size > r2size) {
				while (r2size != 0) {
					r1[r1size] &= r2[r2size];
					r2size--;
					r1size--;
				}
				return r1;
			}
			else {
				while (r1size != 0) {
					r1[r2size] &= r2[r1size];
					r2size--;
					r1size--;
				}
				return r2;
			}
			return r1;
		}
		
		deque<bool> orRs(deque<bool> r1, deque<bool> r2) {
			int r1size = r1.size();
			int r2size = r2.size();
			if (r1size > r2size) {
				while (r2size != 0) {
					r1[r1size] |= r2[r2size];
					r2size--;
					r1size--;
				}
				return r1;
			}
			else {
				while (r1size != 0) {
					r1[r2size] |= r2[r1size];
					r2size--;
					r1size--;
				}
				return r2;
			}
			return r1;
		}
		
		deque<bool> xorRs(deque<bool> r1, deque<bool> r2) {
			int r1size = r1.size();
			int r2size = r2.size();
			if (r1size > r2size) {
				while (r2size != 0) {
					r1[r1size] ^= r2[r2size];
					r2size--;
					r1size--;
				}
				return r1;
			}
			else {
				while (r1size != 0) {
					r1[r2size] ^= r2[r1size];
					r2size--;
					r1size--;
				}
				return r2;
			}
			return r1;
		}
		
		string findInStr(string const& str, int n) {
			if (str.length() < n) {
				return str;
			}
			return str.substr(0, n);
		}
		
		void dequePrint(deque<bool> reg) {
			cout << "deque: {";
				int regSize = reg.size();
				for (int i = 0; i < regSize; i++)
				{
					if (i != regSize - 1)
					cout << reg[i] << ",";
					else
					cout << reg[i] << "}";
			}
			cout << "; size: " << regSize << "." << endl;
			return;
		}
		
		int powmod(int x, int e, int n) {
			for (int i = 0; i < e - 1; i++)
			{
				x = (x * x) % n;
			}
			return x;
		}
		
		long long nod(long long x, long long y) {
			while (x != y) {
				if (x > y) {
					x = x - y;
				}
				else {
					y = y - x;
				}
			}
			return x;
		}
		
		bool mal_fermaPrime(long long p) {
			vector <long long> test;
			long long prime = 0;
			long long not_prime = 0;
			for (long long i = 1; i < 100; i++)
			{
				long long pi = p - i;
				if (pi >= 2 && nod(pi, p) == 1)
				{
					test.push_back(pi);
				}
			}
			for (long long i = 0; i < test.size(); i++)
			{
				long long a_step = test[i];
				long long a_step_dop = test[i];
				for (int j = 1; j < p - 1; j++)
				{
					a_step *= a_step_dop;
					a_step = a_step % p;
				}
				if (a_step == 1) {
					prime++;
				}
				else {
					not_prime++;
				}
			}
			if (prime <= not_prime)
			return false;
			else
			return true;
		}
		
		bool kratP(long long a, long long b) {
			for (int i = 2; i < b; i++)
			{
				if (b % i == 0 && a % i == 0 && mal_fermaPrime(i))
				return true;
			}
			return false;
		}
		
		vector<int> progressVector(int NumCount) {
			vector<int> shows;
			int NumCount10 = NumCount / 10;
			for (int i = 1; i < 10; i++)
			{
				shows.push_back(NumCount10 * i);
			}
			shows.push_back(NumCount - 1);
			cout << "\nThe process of generating pseudo-random numbers: \n\n";
			return shows;
		}
		
		deque <bool> gen_p_lc(int p) {
			deque <bool> ps;
			//Модуль, множитель, приращение, начальное значение
			long long m = 7875;
			long long a = 421;
			long long c = 1663;
			long long x0 = 32112;
			
			for (int i = 0; i < p; i++)
			{
				x0 = (x0 * a + c) % m;
				ps.push_back(x0 % 2);
			}
			return ps;
		}
		
		void helpFunc() {
			cout << "The command with /h is introduced. Permissible parameters:";
			cout << "\n\n/g:<method_code> - the parameter specifies the method of DRB generation, where the method_code can be one of the following:\n";
			cout << "\n  lc - linear congruent method (Input: module, multiplier, increment, initial value);\n add - additive method (Input: module, low index, high index, sequence of initial values);\n 5p - five-parameter method (Input: p, q1, q2, q3, w);\n lfsr - Linear-feedback shift register (LFSR) (Input: binary representation of the coefficient vector, initial value of the register);\n nfsr - non-linear combination of LFSR (Input: binary representation of the coefficient vectors for R1, R2, R3);\n mt - Mersenne vortex (Input: module, initial value of x);\n rc4 - RC4 (Input: 256 initial values);\n rsa - RSA-based PRNG (Input: module n, number e, initial value of x; e satisfies conditions: 1 < e < (p-1)(q-1), NOD(e, (p-1)(q-1)) = 1, where p*q=n. x from the interval [1,n]);\n bbs - Blum-Blum-Schube algorithm (Input: Initial value of x (mutually prime with n));\n";
			cout << "\n\n/i:<number> - initialization vector of the generator.";
			cout << "\n\n/n:<length> - the number of generated numbers. If the parameter is not specified - 10000 numbers are generated.";
			cout << "\n\n/f:<full_file_name> - full name of the file in which the data will be output. If the parameter is not specified, the data should be written to the file named rnd.dat.";
			cout << "\n\n/h - information about the allowed command line parameters of the program.\n";
			
		}
		
		// Linear congruent method
		void lcFunc(string codeMethod, vector <int> genVec, int NumCount, const char* fileName, bool defaultVec) {
			long long m = 12960;
			long long a = 1741;
			long long c = 2731;
			long long x0 = 1;
			if (!defaultVec) {
				m = genVec[0];
				a = genVec[1];
				c = genVec[2];
				x0 = genVec[3];
			}
			if (m <= 0 || a > m || a < 0 || c > m || c < 0 || x0 > m || x0 < 0) {
				cout << "\nError. The parameters do not meet the requirements.\n";
				return;
			}
			if (nod(c, m) != 1) {
				cout << "\nThe parameters do not meet the requirements of Theorem 3.1 (Item 1).\n";
			}
			if (!(kratP(a - 1, m))) {
				cout << "\nThe parameters do not meet the requirements of Theorem 3.1 (Item 2).\n";
			}
			if (m % 4 == 0 && (a - 1) % 4 != 0) {
			cout << "\nThe parameters do not meet the requirements of Theorem 3.1 (Item 3).\n";
		}
		vector<int> showsVector = progressVector(NumCount);
		int check_progress_index = 0;
		
		ofstream f;
		f.open(fileName, ios::out);
		for (int i = 0; i < NumCount; i++)
		{
			x0 = (x0 * a + c) % m;
			if (defaultVec)
			f << x0 % 1001 << ",";
			else
			f << x0 << ",";
			
			if (i == showsVector[check_progress_index]) {
				cout << (check_progress_index + 1) * 10 << "% completed\n";
				check_progress_index++;
			}
		}
		f.close();
		return;
	}
	
	// Additive method
	void addFunc(string codeMethod, vector <int> genVec, int NumCount, const char* fileName, bool defaultVec) {
	// Input: module, low index, high index, sequence of initial values
	long long m = 8001;
	long long k = 18;
	long long j = 65;
	long long x0;
	vector <long long> xNs;
	if (!defaultVec) {
		m = genVec[0];
		k = genVec[1];
		j = genVec[2];
		for (int i = 3; i < genVec.size(); i++)
		{
			xNs.push_back(genVec[i]);
		}
	}
	else {
		xNs = { 816, 159, 798, 290, 168, 441, 691, 655, 874, 220, 125, 977, 586, 381, 868, 294, 948, 437, 581, 181, 701, 536, 11, 672, 103, 601, 794, 189, 12, 130, 386, 828, 288, 183, 117, 456, 624, 807, 110, 498, 27, 234, 474, 613, 615, 341, 906, 562, 778, 486, 155, 276, 894, 441, 226, 762, 234, 762, 98, 458, 399, 445, 765, 223, 879 }
		;
	}
	
	if (k < 1 || k >= j || j < 1) {
		cout << "\nError. The parameters do not meet the requirements.\n";
		return;
	}
	
	vector<int> showsVector = progressVector(NumCount);
	int check_progress_index = 0;
	
	ofstream f;
	f.open(fileName, ios::out);
	long long maxkj = max(k, j);
	if (xNs.size() < maxkj) {
		cout << "\nError. The parameters do not meet the requirements (The transferred initial values are insufficient).\n";
		return;
	}
	for (int i = maxkj + 1; i < NumCount + maxkj + 1; i++)
	{
		x0 = (xNs[i - k] + xNs[i - j]) % m;
		xNs.push_back(x0);
		if (defaultVec)
		f << x0 % 1001 << ",";
		else
		f << x0 << ",";
		
		if (i == showsVector[check_progress_index]) {
			cout << (check_progress_index + 1) * 10 << "% completed\n";
			check_progress_index++;
		}
	}
	f.close();
	return;
	}
	
	// Five-parameter method
	void fivePFunc(string codeMethod, vector <int> genVec, int NumCount, const char* fileName, bool defaultVec) {
	// Input: p, q1, q2, q3, w
	int p = 4253;
	long long q1 = 1093;
	long long q2 = 2254;
	long long q3 = 3297;
	int w = 16;
	if (!defaultVec) {
		p = genVec[0];
		q1 = genVec[1];
		q2 = genVec[2];
		q3 = genVec[3];
		w = genVec[4];
	}
	
	if (q1 >= p || q2 >= p || q3 >= p) {
		cout << "\nError. The parameters do not meet the requirements.\n";
		return;
	}
	
	int regLength = p;
	deque <bool> reg = gen_p_lc(p);
	vector<int> showsVector = progressVector(NumCount);
	int check_progress_index = 0;
	
	ofstream f;
	f.open(fileName, ios::out);
	for (int i = 0; i < NumCount; i++)
	{
		bool xNplusP = reg[regLength - q1 - 1] ^ reg[regLength - q2 - 1] ^ reg[regLength - q3 - 1];
		reg.push_front(xNplusP);
		reg.pop_back();
		long long ReginDec = 0;
		for (int h = 0; h < w; h++)
		{
			ReginDec *= 2;
			ReginDec += reg[h];
		}
		if (defaultVec)
		f << ReginDec % 1001 << ",";
		else
		f << ReginDec << ",";
		
		if (i == showsVector[check_progress_index]) {
			cout << (check_progress_index + 1) * 10 << "% completed\n";
			check_progress_index++;
		}
	}
	f.close();
	return;
	}
	
	// Linear-feedback shift register (LFSR)
	void lfsrFunc(string codeMethod, vector <int> genVec, int NumCount, const char* fileName, bool defaultVec) {
	// Input: binary representation of the coefficient vector, initial value of the register
	if (defaultVec) {
		genVec = { 121, 14, 5, 3, 1, 0 }; // x0, coeffs (100000000101011)
	}
	int x0 = genVec[0];
	int x0Length = log2(x0) + 1;
	int regLength = x0Length - 1;
	bool bigCoeff = false;
	for (int i = 1; i < genVec.size(); i++)
	{
		if (genVec[i] > regLength) {
			regLength = genVec[i] + 1;
			bigCoeff = true;
		}
	}
	if (x0 < 0) {
		cout << "\nError. The parameters do not meet the requirements.\n";
		return;
	}
	
	vector<int> showsVector = progressVector(NumCount);
	int check_progress_index = 0;
	
	deque<bool> reg;
	if (!bigCoeff) {
		regLength++;
		reg = binary(x0, regLength);
	}
	else {
		reg.assign(regLength - x0Length, 0);
		deque<bool> regDop = binary(x0, x0Length);
		for (int i = 0; i < regDop.size(); i++)
		{
			reg.push_back(regDop[i]);
		}
	}
	ofstream f;
	f.open(fileName, ios::out);
	for (int i = 0; i < NumCount; i++)
	{
		bool xNplusP = reg[regLength - genVec[1] - 1];
		for (int j = 2; j < genVec.size(); j++)
		{
			xNplusP ^= reg[regLength - genVec[j] - 1];
		}
		reg.push_front(xNplusP);
		reg.pop_back();
		long long ReginDec = 0;
		for (int h = 0; h < reg.size(); h++)
		{
			ReginDec *= 2;
			ReginDec += reg[h];
		}
		if (defaultVec)
		f << ReginDec % 1001 << ",";
		else
		f << ReginDec << ",";
		
		if (i == showsVector[check_progress_index]) {
			cout << (check_progress_index + 1) * 10 << "% completed\n";
			check_progress_index++;
		}
	}
	f.close();
	return;
	}
	
	// Non-linear combination of LFSR
	void nfsrFunc(string codeMethod, vector <int> genVec, int NumCount, const char* fileName, vector <deque <bool>> nfsrRegs, bool defaultVec) {
	// Input: binary representation of the coefficient vectors for R1, R2, R3
	if (defaultVec) {
		genVec = { 5, 14, 4, 3, 1, 0, 3, 15, 1, 0, 5, 16, 5, 3, 2, 0 }; // size1,coeffs1,size2,coeffs2,size3,coeffs3 (100000000011011,1000000000000011,10000000000101101)
		nfsrRegs = { {1,0,0,0,0,0,0,0,0,0,1,1,0,1,1}, {1,0,0,0,0,0,0,0,0,0,0,0,0,0,1,1}, {1,0,0,0,0,0,0,0,0,0,0,1,0,1,1,0,1} };
	}
	int nachR1 = 1;
	int endR1 = genVec[0] + 1;
	int nachR2 = endR1 + 1;
	int endR2 = genVec[endR1] + nachR2 + 1;
	int nachR3 = endR2 + 1;
	int endR3 = genVec.size();
	vector<int> startsRslos = { nachR1, nachR2, nachR3 };
	vector<int> endssRslos = { endR1, endR2, endR3 };
	vector<int> regLengths;
	for (int i = 0; i < 3; i++)
	{
		regLengths.push_back(nfsrRegs[i].size());
	}
	vector <deque<bool>> regs = nfsrRegs;
	
	vector<int> showsVector = progressVector(NumCount);
	int check_progress_index = 0;
	
	ofstream f;
	f.open(fileName, ios::out);
	for (int i = 0; i < NumCount; i++)
	{
		for (int j = 0; j < 3; j++)
		{
			int startH = startsRslos[j];
			int endH = endssRslos[j];
			bool xNext = regs[j][regLengths[j] - genVec[startH] - 1];
			for (int h = startH; h < endH; h++)
			{
				xNext ^= regs[j][regLengths[j] - genVec[h] - 1];
			}
			regs[j].push_front(xNext);
			regs[j].pop_back();
		}
		vector <deque<bool>> regs2(2);
		regs2[0] = multRs(regs[0], regs[1]);
		regs2[1] = multRs(regs[1], regs[2]);
		deque<bool> resR = xorRs(xorRs(regs2[0], regs2[1]), regs[2]);
		
		long long ReginDec = 0;
		for (int h = 0; h < resR.size(); h++)
		{
			ReginDec *= 2;
			ReginDec += resR[h];
		}
		if (defaultVec)
		f << ReginDec % 1001 << ",";
		else
		f << ReginDec << ",";
		
		if (i == showsVector[check_progress_index]) {
			cout << (check_progress_index + 1) * 10 << "% completed\n";
			check_progress_index++;
		}
	}
	f.close();
	return;
	}
	
	// Mersenne vortex
	void mtFunc(string codeMethod, vector <int> genVec, int NumCount, const char* fileName, bool defaultVec) {
	// Input: module, initial value of x
	int p = 624;
	int w = 32;
	//string uBin = "10000000000000000000000000000000";
	//string hBin = "01111111111111111111111111111111";
	long long uDec = 2147483647;
	long long hDec = 2147483648;
	long long r = 31;
	long long q = 397;
	long long a = 2567483615;
	long long u = 11;
	long long s = 7;
	long long t = 15;
	long long l = 18;
	long long b = 2636928640;
	long long c = 4022730752;
	if (defaultVec) {
		genVec = { 10001, 8191 };
	}
	long long zMod = genVec[0];
	long long x0 = genVec[1];
	vector <long long> x_zns(1, x0);
	for (int i = 0; i < p; i++)
	{
		x_zns.push_back(abs(1812433253 * (x_zns[i - 1] ^ (x_zns[i - 1] >> 30)) + i));
	}
	
	vector<int> showsVector = progressVector(NumCount);
	int check_progress_index = 0;
	
	ofstream f;
	f.open(fileName, ios::out);
	int n = 0;
	for (int countN = 0; countN < NumCount; countN++)
	{
		long long xn = x_zns[n];
		long long xnPlus1 = x_zns[(n + 1) % p];
		long long yNew = (xn & uDec) | (xnPlus1 & hDec);
		int yNewLength = log2(yNew) + 1;
		deque<bool> yBin = binary(yNew, yNewLength);
		long long Xnew;
		yNew >>= 1;
		long long xnqp = x_zns[(n + q) % p];
		if (yBin[yBin.size() - 1] == 1)
		Xnew = xnqp ^ yNew ^ a;
		else
		Xnew = xnqp ^ yNew ^ 0;
		yNew = Xnew;
		yNew = yNew ^ (yNew >> u);
		yNew = yNew ^ ((yNew << s) & b);
		yNew = yNew ^ ((yNew << t) & c);
		long long zNew = yNew ^ (yNew >> l);
		if (defaultVec)
		f << (zNew % zMod) % 1001 << ",";
		else
		f << zNew % zMod << ",";
		
		if (countN == showsVector[check_progress_index]) {
			cout << (check_progress_index + 1) * 10 << "% completed\n";
			check_progress_index++;
		}
		x_zns[n] = Xnew;
		n = (n + 1) % p;
	}
	f.close();
	return;
	}
	
	// RC4
	void rc4Func(string codeMethod, vector <int> genVec, int NumCount, const char* fileName, bool defaultVec) {
	// Input: 256 initial values
	if (defaultVec) {
		genVec = { 802, 720, 341, 337, 961, 882, 417, 785, 198, 727, 899, 372, 374, 425, 556, 615, 813, 768, 840, 183, 893, 568, 73, 387, 18, 436, 182, 125, 806, 899, 485, 607, 619, 825, 944, 579, 707, 360, 363, 904, 87, 262, 276, 460, 687, 831, 75, 499, 599, 915, 681, 492, 483, 754, 878, 500, 189, 60, 624, 994, 959, 109, 600, 577, 934, 544, 156, 640, 903, 519, 544, 990, 781, 819, 449, 468, 650, 524, 967, 248, 438, 647, 739, 920, 400, 617, 419, 588, 676, 43, 581, 634, 151, 181, 211, 84, 724, 367, 723, 627, 886, 267, 617, 667, 85, 65, 134, 735, 589, 100, 983, 26, 747, 721, 945, 147, 337, 364, 734, 13, 406, 315, 647, 556, 496, 858, 640, 220, 224, 362, 847, 110, 629, 463, 776, 713, 528, 909, 448, 116, 9, 430, 141, 755, 151, 86, 901, 488, 449, 635, 500, 855, 950, 147, 410, 446, 4, 49, 665, 227, 411, 511, 336, 39, 974, 112, 752, 501, 21, 200, 617, 29, 629, 757, 784, 779, 843, 684, 266, 292, 319, 766, 146, 269, 912, 556, 714, 916, 605, 378, 142, 15, 889, 478, 54, 862, 590, 806, 363, 610, 5, 979, 638, 634, 736, 421, 413, 578, 105, 679, 869, 424, 444, 14, 692, 356, 569, 405, 271, 173, 783, 413, 188, 671, 891, 242, 533, 480, 48, 895, 89, 53, 873, 727, 686, 608, 147, 98, 185, 252, 776, 54, 675, 220, 67, 366, 576, 636, 771, 846, 808, 553, 259, 996, 224, 149 };
	}
	vector <int> k = genVec;
	vector <int> s = genVec;
	int j = 0;
	for (int i = 0; i < 256; i++)
	{
		j = (j + s[i] + k[i]) % 256;
		swap(s[i], s[j]);
	}
	int i = 0;
	j = 0;
	
	vector<int> showsVector = progressVector(NumCount);
	int check_progress_index = 0;
	
	ofstream f;
	f.open(fileName, ios::out);
	for (int countN = 0; countN < NumCount; countN++)
	{
		i = (i + 1) % 256;
		j = (j + s[j]) % 256;
		swap(s[i], s[j]);
		int t = (s[i] + s[j]) % 256;
		if (defaultVec)
		f << s[t] % 1001 << ",";
		else
		f << s[t] << ",";
		
		if (countN == showsVector[check_progress_index]) {
			cout << (check_progress_index + 1) * 10 << "% completed\n";
			check_progress_index++;
		}
	}
	f.close();
	return;
	}
	
	// RSA-based PRNG
	void rsaFunc(string codeMethod, vector <int> genVec, int NumCount, const char* fileName, bool defaultVec) {
	// Input: module n, number e, initial value of x; e satisfies conditions: 1 < e < (p-1)(q-1), NOD(e, (p-1)(q-1)) = 1, where p*q=n. x from the interval [1,n]
	if (defaultVec) {
		genVec = { 7191817, 151, 69 };
	}
	int n = genVec[0];
	int e = genVec[1];
	int x = genVec[2];
	int l = 20;
	
	if (e <= 1 || x < 1 || x > n - 1) {
		cout << "\nError. The parameters do not meet the requirements.\n";
		return;
	}
	
	vector<int> showsVector = progressVector(NumCount);
	int check_progress_index = 0;
	
	ofstream f;
	f.open(fileName, ios::out);
	for (int countN = 0; countN < NumCount; countN++)
	{
		string zs = "";
		for (int i = 0; i < l; i++)
		{
			x = powmod(x, e, n);
			char z = '0';
			if (x % 2 != 0)
			z = '1';
			zs.push_back(z);
		}
		if (defaultVec)
		f << binToDec(zs) % 1001 << ",";
		else
		f << binToDec(zs) << ",";
		
		if (countN == showsVector[check_progress_index]) {
			cout << (check_progress_index + 1) * 10 << "% completed\n";
			check_progress_index++;
		}
	}
	f.close();
	return;
	}
	
	// Blum-Blum-Schube algorithm
	void bbsFunc(string codeMethod, vector <int> genVec, int NumCount, const char* fileName, bool defaultVec) {
	// Input: Initial value of x (mutually prime with n)
	if (defaultVec) {
		genVec = { 8627 };
	}
	int x0 = genVec[0];
	int n = 16637;
	int l = 10;
	
	if (nod(x0, n) != 1) {
		cout << "\nThe parameters do not meet the requirements (x and n are not mutually prime).\n";
	}
	
	vector<int> showsVector = progressVector(NumCount);
	int check_progress_index = 0;
	
	ofstream f;
	f.open(fileName, ios::out);
	for (int countN = 0; countN < NumCount; countN++)
	{
		x0 = (x0 * x0) % n;
		string zs = "";
		for (int i = 0; i < l; i++)
		{
			x0 = (x0 * x0) % n;
			char z = '0';
			if (x0 % 2 != 0)
			z = '1';
			zs.push_back(z);
		}
		if (defaultVec)
		f << binToDec(zs) % 1001 << ",";
		else
		f << binToDec(zs) << ",";
		
		if (countN == showsVector[check_progress_index]) {
			cout << (check_progress_index + 1) * 10 << "% completed\n";
			check_progress_index++;
		}
	}
	f.close();
	return;
	}
	
	void setMethod(string codeMethod, vector <int> genVec, int NumCount, const char* fileName, int genVecSize, vector <deque <bool>> nfsrRegs, bool defaultVec) {
	if (codeMethod == "lc")
	lcFunc(codeMethod, genVec, NumCount, fileName, defaultVec);
	else if (codeMethod == "add")
	addFunc(codeMethod, genVec, NumCount, fileName, defaultVec);
	else if (codeMethod == "5p")
	fivePFunc(codeMethod, genVec, NumCount, fileName, defaultVec);
	else if (codeMethod == "lfsr")
	lfsrFunc(codeMethod, genVec, NumCount, fileName, defaultVec);
	else if (codeMethod == "nfsr")
	nfsrFunc(codeMethod, genVec, NumCount, fileName, nfsrRegs, defaultVec);
	else if (codeMethod == "mt")
	mtFunc(codeMethod, genVec, NumCount, fileName, defaultVec);
	else if (codeMethod == "rc4")
	rc4Func(codeMethod, genVec, NumCount, fileName, defaultVec);
	else if (codeMethod == "rsa")
	rsaFunc(codeMethod, genVec, NumCount, fileName, defaultVec);
	else if (codeMethod == "bbs")
	bbsFunc(codeMethod, genVec, NumCount, fileName, defaultVec);
	}
	
	int main(int argc, char* argv[])
	{
	setlocale(LC_ALL, "Rus");
	string codeMethod;
	vector <int> genVec;
	vector <deque <bool>> nfsrRegs;
	int NumCount = 10000;
	string fName = "rnd.dat";
	bool defaultVec = true;
	for (int i = 0; argv[i]; i++)
	{
		string checkStr = string(argv[i]);
		if (findInStr(checkStr, 2) == "/h") {
			helpFunc();
			return 0;
		}
		if (checkStr.length() > 2) {
			string ifStr = findInStr(checkStr, 3);
			string subStr = checkStr.substr(3, checkStr.length());
			if (ifStr == "/g:") {
				codeMethod = subStr;
			}
			if (ifStr == "/i:" && (codeMethod == "lfsr" || codeMethod == "nfsr")) {
				char symbol = ',';
				pair<vector<int>, vector <deque <bool>>> nfsrPair;
				nfsrPair = splitStrFsr(subStr, symbol, codeMethod);
				genVec = nfsrPair.first;
				nfsrRegs = nfsrPair.second;
				defaultVec = false;
			}
			else if (ifStr == "/i:") {
				char symbol = ',';
				genVec = splitStr(subStr, symbol);
				defaultVec = false;
			}
			if (ifStr == "/n:") {
				NumCount = stoi(subStr);
			}
			if (ifStr == "/f:") {
				fName = subStr;
			}
		}
	}
	const char* fileName = fName.c_str();
	int genVecSize = genVec.size();
	setMethod(codeMethod, genVec, NumCount, fileName, genVecSize, nfsrRegs, defaultVec);
	cout << "\nGeneration method: " << codeMethod << endl;
	cout << "Count of generated numbers: " << NumCount << endl;
	cout << "Full name of the output file: " << fileName << endl;
	return 0;
	}
	\end{minted}
	
	\section{Исходный код задания 2}
	
	\begin{minted}[fontsize=\footnotesize]{cpp}
		#define _USE_MATH_DEFINES
		
		#include <iostream>
		#include <string>
		#include <vector>
		#include <fstream>
		#include <iomanip>
		#include <bitset>
		#include <sstream>
		#include <deque>
		#include <cmath>
		
		using namespace std;
		
		string findInStr(string const& str, int n) {
			if (str.length() < n) {
				return str;
			}
			return str.substr(0, n);
		}
		
		void helpFunc() {
			cout << "Введена команда c /h. Допустимые параметры:";
			cout << "\n\n/d:<распределение> - код распределения для преобразования последовательности. Возможные коды распределений:\n";
			cout << "\n  st – стандартное равномерное с заданным интервалом (Параметры: a, b);\n  tr – треугольное распределение (Параметры: a, b);\n  ex – общее экспоненциальное распределение (Параметры: a, b);\n  nr – нормальное распределение (Параметры: mu, sigma);\n  gm – гамма распределение (Параметры: a, b);\n  ln – логнормальное распределение (Параметры: a, b);\n  ls – логистическое распределение (Параметры: a, b);\n  bi – биномиальное распределение. (Параметры: n, p)\n";
			cout << "\n\n/f:<имя_файла> - имя файла с входной последовательностью";
			cout << "\n\n/p1:<параметр1> - 1-й параметр, необходимый, для генерации ПСЧ заданного распределения.";
			cout << "\n\n/p2:<параметр2> - 2-й параметр, необходимый, для генерации ПСЧ заданного распределения.";
			cout << "\n\n/h – информация о допустимых параметрах командной строки программы.\n";
		}
		
		vector <double> XtoU(vector <double> allElems) {
			int allElemsSize = allElems.size();
			double max = 2.22507e-308;
			for (int i = 0; i < allElemsSize; i++)
			{
				if (allElems[i] > max)
				max = allElems[i];
			}
			max++;
			for (int i = 0; i < allElemsSize; i++)
			{
				allElems[i] = allElems[i] / max;
			}
			return allElems;
		}
		
		vector <double> stFunc(string distributionMethod, string inputFileName, string parameter1, string parameter2, bool defaultVec, bool& errorMessage) {
			ifstream inFile(inputFileName);
			string lineFile;
			vector <double> allElems;
			if (!inFile.is_open()) {
				cout << "Ошибка открытия файла " << inputFileName << endl;
				errorMessage = true;
				return allElems;
			}
			string getlineElem;
			getline(inFile, getlineElem);
			stringstream ss(getlineElem);
			while (getline(ss, getlineElem, ',')) {
				double intElem = stod(getlineElem);
				allElems.push_back(intElem);
			}
			inFile.close();
			
			vector <double> allU = XtoU(allElems);
			double a = stod(parameter1);
			double b = stod(parameter2);
			for (int i = 0; i < allU.size(); i++)
			{
				allU[i] = (b * allU[i]) + a;
			}
			
			return allU;
		}
		
		vector <double> trFunc(string distributionMethod, string inputFileName, string parameter1, string parameter2, bool defaultVec, bool& errorMessage) {
			ifstream inFile(inputFileName);
			string lineFile;
			vector <double> allElems;
			if (!inFile.is_open()) {
				cout << "Ошибка открытия файла " << inputFileName << endl;
				errorMessage = true;
				return allElems;
			}
			string getlineElem;
			getline(inFile, getlineElem);
			stringstream ss(getlineElem);
			while (getline(ss, getlineElem, ',')) {
				double intElem = stod(getlineElem);
				allElems.push_back(intElem);
			}
			inFile.close();
			
			vector <double> allU = XtoU(allElems);
			double a = stod(parameter1);
			double b = stod(parameter2);
			for (int i = 0; i < allU.size() - 1; i++)
			{
				allU[i] = a + (b * (allU[i] + allU[i + 1] - 1));
			}
			
			return allU;
		}
		
		vector <double> exFunc(string distributionMethod, string inputFileName, string parameter1, string parameter2, bool defaultVec, bool& errorMessage) {
			ifstream inFile(inputFileName);
			string lineFile;
			vector <double> allElems;
			if (!inFile.is_open()) {
				cout << "Ошибка открытия файла " << inputFileName << endl;
				errorMessage = true;
				return allElems;
			}
			string getlineElem;
			getline(inFile, getlineElem);
			stringstream ss(getlineElem);
			while (getline(ss, getlineElem, ',')) {
				double intElem = stod(getlineElem);
				allElems.push_back(intElem);
			}
			inFile.close();
			
			vector <double> allU = XtoU(allElems);
			double a = stod(parameter1);
			double b = stod(parameter2);
			for (int i = 0; i < allU.size() - 1; i++)
			{
				if (allU[i] == 0) {
					cout << "Одно из случайных чисел U = 0. Ошибка в вычислении ln(U)" << endl;
					errorMessage = true;
					return allU;
				}
				allU[i] = ((b * (-1)) * log(allU[i])) + a;
			}
			
			return allU;
		}
		
		vector <double> nrFunc(string distributionMethod, string inputFileName, string parameter1, string parameter2, bool defaultVec, bool& errorMessage) {
			ifstream inFile(inputFileName);
			string lineFile;
			vector <double> allElems;
			if (!inFile.is_open()) {
				cout << "Ошибка открытия файла " << inputFileName << endl;
				errorMessage = true;
				return allElems;
			}
			string getlineElem;
			getline(inFile, getlineElem);
			stringstream ss(getlineElem);
			while (getline(ss, getlineElem, ',')) {
				double intElem = stod(getlineElem);
				allElems.push_back(intElem);
			}
			inFile.close();
			
			if (allElems.size() % 2 != 0)
			allElems.erase(allElems.end() - 1);
			
			vector <double> allU = XtoU(allElems);
			double mu = stod(parameter1);
			double sigma = stod(parameter2);
			for (int i = 0; i < allU.size(); i += 2)
			{
				if (allU[i] == 1) {
					cout << "Одно из случайных чисел U = 1. Ошибка в вычислении ln(1 - U)" << endl;
					errorMessage = true;
					return allU;
				}
				double U1 = allU[i];
				double U2 = allU[i + 1];
				allU[i] = mu + (sigma * sqrt(-2 * log(1 - U1)) * cos(2 * M_PI * U2)); // Z1
				allU[i + 1] = mu + (sigma * sqrt(-2 * log(1 - U1)) * sin(2 * M_PI * U2)); // Z2
			}
			
			return allU;
		}
		
		vector <double> gmFunc(string distributionMethod, string inputFileName, string parameter1, string parameter2, bool defaultVec, bool& errorMessage) {
			ifstream inFile(inputFileName);
			string lineFile;
			vector <double> allElems;
			if (!inFile.is_open()) {
				cout << "Ошибка открытия файла " << inputFileName << endl;
				errorMessage = true;
				return allElems;
			}
			string getlineElem;
			getline(inFile, getlineElem);
			stringstream ss(getlineElem);
			while (getline(ss, getlineElem, ',')) {
				double intElem = stod(getlineElem);
				allElems.push_back(intElem);
			}
			inFile.close();
			
			vector <double> allU = XtoU(allElems);
			double a = stod(parameter1);
			double b = stod(parameter2);
			for (int i = 0; i < allU.size(); i++)
			{
				if (allU[i] == 1) {
					cout << "Одно из случайных чисел U = 1. Ошибка в вычислении ln(1 - U)" << endl;
					errorMessage = true;
					return allU;
				}
				allU[i] = a - (b * log(1 - allU[i]));
			}
			
			return allU;
		}
		
		vector <double> lnFunc(string distributionMethod, string inputFileName, string parameter1, string parameter2, bool defaultVec, bool& errorMessage) {
			ifstream inFile(inputFileName);
			string lineFile;
			vector <double> allElems;
			if (!inFile.is_open()) {
				cout << "Ошибка открытия файла " << inputFileName << endl;
				errorMessage = true;
				return allElems;
			}
			string getlineElem;
			getline(inFile, getlineElem);
			stringstream ss(getlineElem);
			while (getline(ss, getlineElem, ',')) {
				double intElem = stod(getlineElem);
				allElems.push_back(intElem);
			}
			inFile.close();
			
			if (allElems.size() % 2 != 0)
			allElems.erase(allElems.end() - 1);
			
			vector <double> allU = XtoU(allElems);
			double a = stod(parameter1);
			double b = stod(parameter2);
			for (int i = 0; i < allU.size(); i += 2)
			{
				if (allU[i] == 1) {
					cout << "Одно из случайных чисел U = 1. Ошибка в вычислении ln(1 - U)" << endl;
					errorMessage = true;
					return allU;
				}
				double U1 = allU[i];
				double U2 = allU[i + 1];
				double Z1 = a + (b * sqrt(-2 * log(1 - U1)) * cos(2 * M_PI * U2));
				double Z2 = a + (b * sqrt(-2 * log(1 - U1)) * sin(2 * M_PI * U2));
				double bminZ1 = b - Z1;
				if (bminZ1 > 43) {
					cout << "Разность b - Z1 больше 43. Выход за границы double в вычислении exp(b - Z1)" << endl;
					errorMessage = true;
					return allU;
				}
				double bminZ2 = b - Z2;
				if (bminZ1 > 43) {
					cout << "Разность b - Z2 больше 43. Выход за границы double в вычислении exp(b - Z2)" << endl;
					errorMessage = true;
					return allU;
				}
				allU[i] = a + exp(bminZ1);
				allU[i + 1] = a + exp(bminZ2);
			}
			
			return allU;
		}
		
		vector <double> lsFunc(string distributionMethod, string inputFileName, string parameter1, string parameter2, bool defaultVec, bool& errorMessage) {
			ifstream inFile(inputFileName);
			string lineFile;
			vector <double> allElems;
			if (!inFile.is_open()) {
				cout << "Ошибка открытия файла " << inputFileName << endl;
				errorMessage = true;
				return allElems;
			}
			string getlineElem;
			getline(inFile, getlineElem);
			stringstream ss(getlineElem);
			while (getline(ss, getlineElem, ',')) {
				double intElem = stod(getlineElem);
				allElems.push_back(intElem);
			}
			inFile.close();
			
			vector <double> allU = XtoU(allElems);
			double a = stod(parameter1);
			double b = stod(parameter2);
			for (int i = 0; i < allU.size(); i += 2)
			{
				if (allU[i] == 1) {
					cout << "Одно из случайных чисел U = 1. Ошибка в вычислении U/(1 - U). Деление на 0." << endl;
					errorMessage = true;
					return allU;
				}
				else if (allU[i] == 0) {
					cout << "Одно из случайных чисел U = 0. Ошибка в вычислении ln(U/(1 - U))" << endl;
					errorMessage = true;
					return allU;
				}
				allU[i] = a + (b * log(allU[i] / (1 - allU[i])));
			}
			
			return allU;
		}
		
		double Cnk(double n, double k) {
			double nfact = 1;
			for (int i = 1; i <= n; i++) {
				nfact = nfact * i;
			}
			double nminkfact = 1;
			for (int i = 1; i <= n - k; i++) {
				nminkfact = nminkfact * i;
			}
			double kfact = 1;
			for (int i = 1; i <= k; i++) {
				kfact = kfact * i;
			}
			double res = nfact / (nminkfact * kfact);
			return res;
		}
		
		double funcBiF(double y, double n, double p) {
			double resSum = 0;
			for (int k = 0; k <= y; k++)
			{
				resSum = resSum + (Cnk(n, k) * pow(p, k) * pow(1 - p, n - k));
			}
			return resSum;
		}
		
		vector <double> biFunc(string distributionMethod, string inputFileName, string parameter1, string parameter2, bool defaultVec, bool& errorMessage) {
			ifstream inFile(inputFileName);
			string lineFile;
			vector <double> allElems;
			if (!inFile.is_open()) {
				cout << "Ошибка открытия файла " << inputFileName << endl;
				errorMessage = true;
				return allElems;
			}
			string getlineElem;
			getline(inFile, getlineElem);
			stringstream ss(getlineElem);
			while (getline(ss, getlineElem, ',')) {
				double intElem = stod(getlineElem);
				allElems.push_back(intElem);
			}
			inFile.close();
			
			vector <double> allU = XtoU(allElems);
			double n = stod(parameter1); //n
			double p = stod(parameter2); //p
			if (p > 1 || p < 0) {
				cout << "Ошибка. Параметр p2 не может быть > 1 или < 0, т.к. это вероятность." << endl;
				errorMessage = true;
				return allElems;
			}
			for (int i = 0; i < allU.size(); i++)
			{
				double y = 0;
				for (int j = 0; j <= n; j++)
				{
					if (allU[i] > funcBiF(y, n, p)) {
						y++;
						if (j == n) {
							cout << "Ошибка. Не нашлось такого y = (0,...,n), при котором U <= F(y)" << endl;
							errorMessage = true;
							return allElems;
						}
					}
					else
					break;
				}
				allU[i] = y;
			}
			
			return allU;
		}
		
		vector <double> setMethod(string distributionMethod, string inputFileName, string parameter1, string parameter2, bool defaultVec, bool& errorMessage) {
			vector <double> distributionResVec;
			if (distributionMethod == "st")
			distributionResVec = stFunc(distributionMethod, inputFileName, parameter1, parameter2, defaultVec, errorMessage);
			else if (distributionMethod == "tr")
			distributionResVec = trFunc(distributionMethod, inputFileName, parameter1, parameter2, defaultVec, errorMessage);
			else if (distributionMethod == "ex")
			distributionResVec = exFunc(distributionMethod, inputFileName, parameter1, parameter2, defaultVec, errorMessage);
			else if (distributionMethod == "nr")
			distributionResVec = nrFunc(distributionMethod, inputFileName, parameter1, parameter2, defaultVec, errorMessage);
			else if (distributionMethod == "gm")
			distributionResVec = gmFunc(distributionMethod, inputFileName, parameter1, parameter2, defaultVec, errorMessage);
			else if (distributionMethod == "ln")
			distributionResVec = lnFunc(distributionMethod, inputFileName, parameter1, parameter2, defaultVec, errorMessage);
			else if (distributionMethod == "ls")
			distributionResVec = lsFunc(distributionMethod, inputFileName, parameter1, parameter2, defaultVec, errorMessage);
			else if (distributionMethod == "bi")
			distributionResVec = biFunc(distributionMethod, inputFileName, parameter1, parameter2, defaultVec, errorMessage);
			
			return distributionResVec;
		}
		
		int main(int argc, char* argv[])
		{
			/*
			Программа для преобразования последовательности ПСЧ
			в другую последовательность ПСЧ с заданным распределением.
			*/
			setlocale(LC_ALL, "Rus");
			string distributionMethod;
			string inputFileName = "rncInput.dat";
			string parameter1;
			string parameter2;
			bool defaultVec = true;
			for (int i = 0; argv[i]; i++)
			{
				string checkStr = string(argv[i]);
				if (findInStr(checkStr, 2) == "/h") {
					helpFunc();
					return 0;
				}
				if (checkStr.length() > 2) {
					string ifStr = findInStr(checkStr, 3);
					string subStr = checkStr.substr(3, checkStr.length());
					if (ifStr == "/d:") {
						distributionMethod = subStr;
					}
					if (ifStr == "/f:") {
						inputFileName = subStr;
					}
					if (ifStr == "/p1") {
						parameter1 = checkStr.substr(4, checkStr.length());
					}
					if (ifStr == "/p2") {
						parameter2 = checkStr.substr(4, checkStr.length());
					}
					
				}
			}
			bool errorMessage = false;
			vector <double> rncRes = setMethod(distributionMethod, inputFileName, parameter1, parameter2, defaultVec, errorMessage);
			if (errorMessage)
			return 0;
			string outputFileName = distributionMethod + "Output.dat";
			ofstream outFile(outputFileName);
			for (int i = 0; i < rncRes.size(); i++)
			{
				if (i == rncRes.size() - 1)
				outFile << (long long)rncRes[i];
				else
				outFile << (long long)rncRes[i] << ",";
			}
			outFile.close();
			cout << "\nDistribution method: " << distributionMethod << endl;
			cout << "Parameter 1: " << parameter1 << endl;
			cout << "Parameter 2: " << parameter2 << endl;
			cout << "Input File: " << inputFileName << endl;
			cout << "Output File: " << outputFileName << endl;
			return 0;
		}
	\end{minted}
	%\includepdf[pages=-]{pril-A.pdf}

	%\includepdf[pages=-]{pril-B.pdf}
	
\end{document}